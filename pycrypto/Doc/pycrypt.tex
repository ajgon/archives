\documentclass{howto}

\title{Python Cryptography Toolkit}

\release{1.9alpha6}

\author{A.M. Kuchling}
\authoraddress{\url{www.amk.ca}}

\begin{document}
\maketitle

\begin{abstract}
\noindent
The Python Cryptography Toolkit describes a package containing various
cryptographic modules for the Python programming language.  This
documentation assumes you have some basic knowledge about the Python
language, but not necessarily about cryptography.

\end{abstract}

\tableofcontents


%======================================================================
\section{Introduction}

\subsection{Design Goals}
The Python cryptography toolkit is intended to provide a reliable and
stable base for writing Python programs that require cryptographic
functions.

A central goal of the author's has been to provide a simple,
consistent interface for similar classes of algorithms.  For example,
all block cipher objects have the same methods and return values, and
support the same feedback modes.  Hash functions have a different
interface, but it too is consistent over all the hash functions
available.  Some of these interfaces have been codified as Python
Enhancement Proposal documents, as \pep{247}, ``API for Cryptographic
Hash Functions'', and \pep{272}, ``API for Block Encryption
Algorithms''.  

This is intended to make it easy to replace old algorithms with newer,
more secure ones.  If you're given a bit of portably-written Python
code that uses the DES encryption algorithm, you should be able to use
AES instead by simply changing \code{from Crypto.Cipher import DES} to
\code{from Crypto.Cipher import AES}, and changing all references to
\code{DES.new()} to \code{AES.new()}.  It's also fairly simple to
write your own modules that mimic this interface, thus letting you use
combinations or permutations of algorithms.

Some modules are implemented in C for performance; others are written
in Python for ease of modification.  Generally, low-level functions
like ciphers and hash functions are written in C, while less
speed-critical functions have been written in Python.  This division
may change in future releases.  When speeds are quoted in this
document, they were measured on a 500 MHz Pentium II running Linux.
The exact speeds will obviously vary with different machines,
different compilers, and the phase of the moon, but they provide a
crude basis for comparison.  Currently the cryptographic
implementations are acceptably fast, but not spectacularly good.  I
welcome any suggestions or patches for faster code.

I have placed the code under no restrictions; you can redistribute the
code freely or commercially, in its original form or with any
modifications you make, subject to whatever local laws may apply in your
jurisdiction.  Note that you still have to come to some agreement with
the holders of any patented algorithms you're using.  If you're
intensively using these modules, please tell me about it; there's little
incentive for me to work on this package if I don't know of anyone using
it.

I also make no guarantees as to the usefulness, correctness, or legality
of these modules, nor does their inclusion constitute an endorsement of
their effectiveness.  Many cryptographic algorithms are patented;
inclusion in this package does not necessarily mean you are allowed to
incorporate them in a product and sell it.  Some of these algorithms may
have been cryptanalyzed, and may no longer be secure.  While I will
include commentary on the relative security of the algorithms in the
sections entitled "Security Notes", there may be more recent analyses
I'm not aware of.  (Or maybe I'm just clueless.)  If you're implementing
an important system, don't just grab things out of a toolbox and put
them together; do some research first.  On the other hand, if you're
just interested in keeping your co-workers or your relatives out of your
files, any of the components here could be used.

This document is very much a work in progress.  If you have any
questions, comments, complaints, or suggestions, please send them to me.

\subsection{Acknowledgements}
Much of the code that actually implements the various cryptographic
algorithms was not written by me.  I'd like to thank all the people who
implemented them, and released their work under terms which allowed me
to use their code.  These individuals are credited in the relevant
chapters of this documentation.  Bruce Schneier's book \emph{Applied
Cryptography} was also very useful in writing this toolkit; I highly
recommend it if you're interested in learning more about cryptography.

Good luck with your cryptography hacking!

A.M.K.

\email{akuchlin@mems-exchange.org}

Washington DC, USA

June 2002


%======================================================================
\section{Crypto.Hash: Hash Functions}

Hash functions take arbitrary strings as input, and produce an output
of fixed size that is dependent on the input; it should never be
possible to derive the input data given only the hash function's
output.  One simple hash function consists of simply adding together
all the bytes of the input, and taking the result modulo 256.  For a
hash function to be cryptographically secure, it must be very
difficult to find two messages with the same hash value, or to find a
message with a given hash value.  The simple additive hash function
fails this criterion miserably and the hash functions described below
meet this criterion (as far as we know).  Examples of
cryptographically secure hash functions include MD2, MD5, and SHA.

Hash functions can be used simply as a checksum, or, in association with a
public-key algorithm, can be used to implement digital signatures.
 
The hashing algorithms currently implemented are:

\begin{tableii}{c|l}{}{Hash function}{Digest length}
\lineii{MD2}{128 bits}
\lineii{MD4}{128 bits}
\lineii{MD5}{128 bits}
\lineii{RIPEMD}{160 bits}
\lineii{SHA}{160 bits}
\end{tableii}

All hashing modules share the same interface.  After importing a given
hashing module, call the \function{new()} function to create a new
hashing object. You can now feed arbitrary strings into the object
with the \method{update()} method, and can ask for the hash value at
any time by calling the \method{digest()} or \method{hexdigest()}
methods.  The \function{new()} function can also be passed an optional
string parameter that will be immediately hashed into the object's
state.

Hash function modules define one variable:

\begin{datadesc}{digest_size}
An integer value; the size of the digest
produced by the hashing objects.  You could also obtain this value by
creating a sample object, and taking the length of the digest string
it returns, but using \member{digest_size} is faster.
\end{datadesc}

The methods for hashing objects are always the following:

\begin{methoddesc}{copy}{}
Return a separate copy of this hashing object.  An \code{update} to
this copy won't affect the original object.
\end{methoddesc}

\begin{methoddesc}{digest}{}
Return the hash value of this hashing object, as a string containing
8-bit data.  The object is not altered in any way by this function;
you can continue updating the object after calling this function.
\end{methoddesc}

\begin{methoddesc}{hexdigest}{}
Return the hash value of this hashing object, as a string containing
the digest data as hexadecimal digits.  The resulting string will be
twice as long as that returned by \method{digest()}.  The object is not
altered in any way by this function; you can continue updating the
object after calling this function.
\end{methoddesc}

\begin{methoddesc}{update}{arg}
Update this hashing object with the string \var{arg}.
\end{methoddesc}

Here's an example, using the MD5 algorithm:

\begin{verbatim}
>>> from Crypto.Hash import MD5
>>> m = MD5.new()
>>> m.update('abc')
>>> m.digest()
'\x90\x01P\x98<\xd2O\xb0\xd6\x96?}(\xe1\x7fr'
>>> m.hexdigest()
'900150983cd24fb0d6963f7d28e17f72'
\end{verbatim}


\subsection{Security Notes}

Hashing algorithms are broken by developing an algorithm to compute a
string that produces a given hash value, or to find two messages that
produce the same hash value. Consider an example where Alice and Bob
are using digital signatures to sign a contract.  Alice computes the
hash value of the text of the contract and signs the hash value with
her private key.  Bob could then compute a different contract that has
the same hash value, and it would appear that Alice signed that bogus
contract; she'd have no way to prove otherwise.  Finding such a
message by brute force takes \code{pow(2, b-1)} operations, where the
hash function produces \emph{b}-bit hashes.

If Bob can only find two messages with the same hash value but can't
choose the resulting hash value, he can look for two messages with
different meanings, such as "I will mow Bob's lawn for $10" and "I owe
Bob $1,000,000", and ask Alice to sign the first, innocuous contract.
This attack is easier for Bob, since finding two such messages by brute
force will take \code{pow(2, b/2)} operations on average.  However,
Alice can protect herself by changing the protocol; she can simply
append a random string to the contract before hashing and signing it;
the random string can then be kept with the signature.

None of the algorithms implemented here have been completely broken.
There are no attacks on MD2, but it's rather slow at 1250 K/sec.  MD4
is faster at 44,500 K/sec but there have been some partial attacks on
it.  MD4 makes three iterations of a basic mixing operation; two of
the three rounds have been cryptanalyzed, but the attack can't be
extended to the full algorithm.  MD5 is a strengthened version of MD4
with four rounds; an attack against one round has been found XXX
update this.  MD5 is still believed secure at the moment, but people
are gravitating toward using SHA in new software because there are no
known attacks against SHA.  The MD5 implementation is moderately
well-optimized and thus faster on x86 processors, running at 35,500
K/sec.  MD5 may even be faster than MD4, depending on the processor
and compiler you use.

All the MD\var{n} algorithms produce 128-bit hashes; SHA produces a
larger 160-bit hash, and there are no known attacks against it.  The
first version of SHA had a weakness which was later corrected; the
code used here implements the second, corrected, version.  It operates
at 21,000 K/sec.  RIPEMD also has a 160-bit output, and operates at
17,600 K/sec.

\subsection{Credits}
The MD2 and  MD4 implementations were written by A.M. Kuchling,
and the MD5 code was implemented by Colin Plumb.  The SHA code was
originally written by Peter Gutmann.  The RIPEMD code was written by
Antoon Bosselaers, and adapted for the toolkit by Hirendra Hindocha.


%======================================================================
\section{Crypto.Cipher: Encryption Algorithms}

Encryption algorithms transform their input data, or \dfn{plaintext},
in some way that is dependent on a variable \dfn{key}, producing
\dfn{ciphertext}. This transformation can easily be reversed, if (and,
hopefully, only if) one knows the key.  The key can be varied by the
user or application and chosen from some very large space of possible
keys.

For a secure encryption algorithm, it should be very difficult to
determine the original plaintext without knowing the key; usually, no
clever attacks on the algorithm are known, so the only way of breaking
the algorithm is to try all possible keys. Since the number of possible
keys is usually of the order of 2 to the power of 56 or 128, this is not
a serious threat, although 2 to the power of 56 is now considered
insecure in the face of custom-built parallel computers and distributed
key guessing efforts.

\dfn{Block ciphers} take multibyte inputs of a fixed size
(frequently 8 or 16 bytes long) and encrypt them.  Block ciphers can
be operated in various modes.  The simplest is Electronic Code Book
(or ECB) mode.  In this mode, each block of plaintext is simply
encrypted to produce the ciphertext.  This mode can be dangerous,
because many files will contain patterns greater than the block size;
for example, the comments in a C program may contain long strings of
asterisks intended to form a box.  All these identical blocks will
encrypt to identical ciphertext; an adversary may be able to use this
structure to obtain some information about the text.

To eliminate this weakness, there are various feedback modes in which
the plaintext is combined with the previous ciphertext before
encrypting; this eliminates any repetitive structure in the
ciphertext.   

One mode is Cipher Block Chaining (CBC mode); another is Cipher
FeedBack (CFB mode).  CBC mode still encrypts in blocks, and thus is
only slightly slower than ECB mode.  CFB mode encrypts on a
byte-by-byte basis, and is much slower than either of the other two
modes.  The chaining feedback modes require an initialization value to
start off the encryption; this is a string of the same length as the
ciphering algorithm's block size, and is passed to the \code{new()}
function.  There is also a special PGP mode, which is an oddball
variant of CFB used by the PGP program.  While you can use it in
non-PGP programs, it's quite non-standard.

The currently available block ciphers are listed in the following table,
and are in the \code{Crypto.Cipher} package:

\begin{tableii}{c|l}{}{Cipher}{Key Size/Block Size}
\lineii{AES}{16, 24, or 32 bytes/16 bytes}
\lineii{ARC2}{Variable/8 bytes}
\lineii{Blowfish}{Variable/8 bytes}
\lineii{CAST}{Variable/8 bytes}
\lineii{DES}{8 bytes/8 bytes}
\lineii{DES3 (Triple DES)}{16 bytes/8 bytes}
\lineii{IDEA}{16 bytes/8 bytes}
\lineii{RC5}{Variable/8 bytes}
\end{tableii}

In a strict formal sense, \dfn{stream ciphers} encrypt data bit-by-bit;
practically, stream ciphers work on a character-by-character basis.
Stream ciphers use exactly the
same interface as block ciphers, with a block length that will always
be 1; this is how block and stream ciphers can be distinguished. 
The only feedback mode available for stream ciphers is ECB mode. 

The currently available stream ciphers are listed in the following table:

\begin{tableii}{c|l}{}{Cipher}{Key Size}
\lineii{Cipher}{Key Size}
  \lineii{ARC4}{Variable}
  \lineii{XOR}{Variable}
\end{tableii}

ARC4 is short for `Alleged RC4'.  In September of 1994, someone posted
C code to both the Cypherpunks mailing list and to the Usenet
newsgroup \code{sci.crypt}, claiming that it implemented the RC4
algorithm.  This claim turned out to be correct.  Note that there's a
damaging class of weak RC4 keys; this module won't warn you about such keys.
% XXX other analyses of RC4?

A similar anonymous posting was made for Alleged RC2 in January, 1996.

An example usage of the DES module:
\begin{verbatim}
>>> from Crypto.Cipher import DES
>>> obj=DES.new('abcdefgh', DES.ECB)
>>> plain="Guido van Rossum is a space alien."
>>> len(plain)
34
>>> obj.encrypt(plain)
Traceback (innermost last):
  File "<stdin>", line 1, in ?
ValueError: Strings for DES must be a multiple of 8 in length
>>> ciph=obj.encrypt(plain+'XXXXXX')
>>> ciph
'\021,\343Nq\214DY\337T\342pA\372\255\311s\210\363,\300j\330\250\312\347\342I\3215w\03561\303dgb/\006'
>>> obj.decrypt(ciph)
'Guido van Rossum is a space alien.XXXXXX'
\end{verbatim}

All cipher algorithms share a common interface.  After importing a
given module, there is exactly one function and two variables
available.

\begin{funcdesc}{new}{key, mode\optional{, IV}}
Returns a ciphering object, using \var{key} and feedback mode
\var{mode}.  If \var{mode} is CBC or CFB, \var{IV} must be provided,
and must be a string of the same length as the block size.  Some
algorithms support additional keyword arguments to this function; see
the "Algorithm-specific Notes for Encryption Algorithms" section below for the details.
\end{funcdesc}

\begin{datadesc}{block_size}
An integer value; the size of the blocks encrypted by this module.
Strings passed to the \code{encrypt} and \code{decrypt} functions
must be a multiple of this length.  For stream ciphers,
\code{block_size} will be 1. 
\end{datadesc}

\begin{datadesc}{key_size}
An integer value; the size of the keys required by this module.  If
\code{key_size} is zero, then the algorithm accepts arbitrary-length
keys.  You cannot pass a key of length 0 (that is, the null string
\code{''} as such a variable-length key.  
\end{datadesc}

All cipher objects have at least three attributes:

\begin{memberdesc}{block_size}
An integer value equal to the size of the blocks encrypted by this object.
Identical to the module variable of the same name.
\end{memberdesc}

\begin{memberdesc}{IV}
Contains the initial value which will be used to start a cipher
feedback mode.  After encrypting or decrypting a string, this value
will reflect the modified feedback text; it will always be one block
in length.  It is read-only, and cannot be assigned a new value.
\end{memberdesc}

\begin{memberdesc}{key_size}
An integer value equal to the size of the keys used by this object.  If
\code{key_size} is zero, then the algorithm accepts arbitrary-length
keys.  For algorithms that support variable length keys, this will be 0.
Identical to the module variable of the same name.  
\end{memberdesc}

All ciphering objects have the following methods:

\begin{methoddesc}{decrypt}{string}
Decrypts \var{string}, using the key-dependent data in the object, and
with the appropriate feedback mode.  The string's length must be an exact
multiple of the algorithm's block size.  Returns a string containing
the plaintext.
\end{methoddesc}

\begin{methoddesc}{encrypt}{string}
Encrypts a non-null \var{string}, using the key-dependent data in the
object, and with the appropriate feedback mode.  The string's length
must be an exact multiple of the algorithm's block size; for stream
ciphers, the string can be of any length.  Returns a string containing
the ciphertext.
\end{methoddesc}


\subsection{Algorithm-specific Notes for Encryption Algorithms}

RC5 has a bunch of parameters; see Ronald Rivest's paper at
\url{http://theory.lcs.mit.edu/~rivest/rc5rev.ps} for the
implementation details.  The keyword parameters are:

\begin{itemize}
\item \code{version}:
The version
of the RC5 algorithm to use; currently the only legal value is
\code{0x10} for RC5 1.0.  
\item \code{wordsize}:
The word size to use;
16 or 32 are the only legal values.  (A larger word size is better, so
usually 32 will be used.  16-bit RC5 is probably only of academic
interest.)  
\item \code{rounds}:
The number of rounds to apply, the larger the more secure: this
can be any value from 0 to 255, so you will have to choose a value
balanced between speed and security. 
\end{itemize}


\subsection{Security Notes}
Encryption algorithms can be broken in several ways.  If you have some
ciphertext and know (or can guess) the corresponding plaintext, you can
simply try every possible key in a \dfn{known-plaintext} attack.  Or, it
might be possible to encrypt text of your choice using an unknown key;
for example, you might mail someone a message intending it to be
encrypted and forwarded to someone else.  This is a
\dfn{chosen-plaintext} attack, which is particularly effective if it's
possible to choose plaintexts that reveal something about the key when
encrypted.

DES (5100 K/sec) has a 56-bit key; this is starting to become too small
for safety.  It has been estimated that it would only cost \$1,000,000 to
build a custom DES-cracking machine that could find a key in 3 hours.  A
chosen-ciphertext attack using the technique of \dfn{linear
cryptanalysis} can break DES in \code{pow(2, 43)} steps.  However,
unless you're encrypting data that you want to be safe from major
governments, DES will be fine. DES3 (1830 K/sec) uses three DES
encryptions for greater security and a 112-bit or 168-bit key, but is
correspondingly slower.

There are no publicly known attacks against IDEA (3050 K/sec), and
it's been around long enough to have been examined.  There are no
known attacks against ARC2 (2160 K/sec), ARC4 (8830 K/sec), Blowfish
(9250 K/sec), CAST (2960 K/sec), or RC5 (2060 K/sec), but they're all
relatively new algorithms and there hasn't been time for much analysis
to be performed; use them for serious applications only after careful
research.

AES, the Advanced Encryption Standard, was chosen by the US National
Institute of Standards and Technology from among 6 competitors, and is
probably your best choice.  It runs at 7060 K/sec, so it's among the
faster algorithms around.


\subsection{Credits}
The code for Blowfish was written by Bryan Olson, partially based on a
previous implementation by Bruce Schneier, who also invented the
algorithm; the Blowfish algorithm has been placed in the public domain
and can be used freely.  (See \url{http://www.counterpane.com} for more
information about Blowfish.)  The CAST implementation was written by 
Wim Lewis.  The DES implementation was written by Eric Young, and the
IDEA implementation by Colin Plumb. The RC5 implementation
was written by A.M. Kuchling.

The Alleged RC4 code was posted to the \code{sci.crypt} newsgroup by an
unknown party, and re-implemented by A.M. Kuchling.  


%======================================================================
\section{Crypto.Protocol: Various Protocols}

\subsection{Crypto.Protocol.AllOrNothing}

This module implements all-or-nothing package transformations.
An all-or-nothing package transformation is one in which some text is
transformed into message blocks, such that all blocks must be obtained before
the reverse transformation can be applied.  Thus, if any blocks are corrupted
or lost, the original message cannot be reproduced.

An all-or-nothing package transformation is not encryption, although a block
cipher algorithm is used.  The encryption key is randomly generated and is
extractable from the message blocks.

\begin{classdesc}{AllOrNothing}{ciphermodule, mode=None, IV=None}
Class implementing the All-or-Nothing package transform.

\var{ciphermodule} is a module implementing the cipher algorithm to
use.  Optional arguments \var{mode} and \var{IV} are passed directly
through to the \var{ciphermodule}.\code{new()} method; they are the
feedback mode and initialization vector to use.  All three arguments
must be the same for the object used to create the digest, and to
undigest'ify the message blocks.

The module passed as \var{ciphermodule} must provide the \pep{272}
interface.  An encryption key is randomly generated automatically when
needed.
\end{classdesc}

The methods of the \class{AllOrNothing} class are:

\begin{methoddesc}{digest}{text}
Perform the All-or-Nothing package transform on the 
string \var{text}.  Output is a list of message blocks describing the
transformed text, where each block is a string of bit length equal
to the cipher module's block_size.
\end{methoddesc}

\begin{methoddesc}{undigest}{mblocks}
Perform the reverse package transformation on a list of message
blocks.  Note that the cipher module used for both transformations
must be the same.  \var{mblocks} is a list of strings of bit length
equal to \var{ciphermodule}'s block_size.  The output is a string object.
\end{methoddesc}


\subsection{Crypto.Protocol.Chaffing}

Winnowing and chaffing is a technique for enhancing privacy without requiring
strong encryption.  In short, the technique takes a set of authenticated
message blocks (the wheat) and adds a number of chaff blocks which have
randomly chosen data and MAC fields.  This means that to an adversary, the
chaff blocks look as valid as the wheat blocks, and so the authentication
would have to be performed on every block.  By tailoring the number of chaff
blocks added to the message, the sender can make breaking the message
computationally infeasible.  There are many other interesting properties of
the winnow/chaff technique.

For example, say Alice is sending a message to Bob.  She packetizes the
message and performs an all-or-nothing transformation on the packets.  Then
she authenticates each packet with a message authentication code (MAC).  The
MAC is a hash of the data packet, and there is a secret key which she must
share with Bob (key distribution is an exercise left to the reader).  She then
adds a serial number to each packet, and sends the packets to Bob.

Bob receives the packets, and using the shared secret authentication key,
authenticates the MACs for each packet.  Those packets that have bad MACs are
simply discarded.  The remainder are sorted by serial number, and passed
through the reverse all-or-nothing transform.  The transform means that an
eavesdropper (say Eve) must acquire all the packets before any of the data can
be read.  If even one packet is missing, the data is useless.

There's one twist: by adding chaff packets, Alice and Bob can make Eve's job
much harder, since Eve now has to break the shared secret key, or try every
combination of wheat and chaff packet to read any of the message.  The cool
thing is that Bob doesn't need to add any additional code; the chaff packets
are already filtered out because their MACs don't match (in all likelihood --
since the data and MACs for the chaff packets are randomly chosen it is
possible, but very unlikely that a chaff MAC will match the chaff data).  And
Alice need not even be the party adding the chaff!  She could be completely
unaware that a third party, say Charles, is adding chaff packets to her
messages as they are transmitted.

\begin{classdesc}{Chaff}{factor=1.0, blocksper=1}
Class implementing the chaff adding algorithm. 
\var{factor} is the number of message blocks 
            to add chaff to, expressed as a percentage between 0.0 and 1.0; the default value is 1.0.
\var{blocksper} is the number of chaff blocks to include for each block
            being chaffed, and defaults to 1.  The default settings 
add one chaff block to every
            message block.  By changing the defaults, you can adjust how
            computationally difficult it could be for an adversary to
            brute-force crack the message.  The difficulty is expressed as:

\begin{verbatim}
pow(blocksper, int(factor * number-of-blocks))
\end{verbatim}

For ease of implementation, when \var{factor} < 1.0, only the first
\code{int(\var{factor}*number-of-blocks)} message blocks are chaffed.
\end{classdesc}

\class{Chaff} instances have the following methods:

\begin{methoddesc}{chaff}{blocks}
Add chaff to message blocks.  \var{blocks} is a list of 3-tuples of the
form (\var{serial-number}, \var{data}, \var{MAC}).

Chaff is created by choosing a random number of the same
byte-length as \var{data}, and another random number of the same
byte-length as \var{MAC}.  The message block's serial number is placed
on the chaff block and all the packet's chaff blocks are randomly
interspersed with the single wheat block.  This method then
returns a list of 3-tuples of the same form.  Chaffed blocks will
contain multiple instances of 3-tuples with the same serial
number, but the only way to figure out which blocks are wheat and
which are chaff is to perform the MAC hash and compare values.
\end{methoddesc}


%======================================================================
\section{Crypto.PublicKey: Public-Key Algorithms}
So far, the encryption algorithms described have all been \dfn{private
key} ciphers.  The same key is used for both encryption and decryption
so all correspondents must know it.  This poses a problem: you may
want encryption to communicate sensitive data over an insecure
channel, but how can you tell your correspondent what the key is?  You
can't just e-mail it to her because the channel is insecure.  One
solution is to arrange the key via some other way: over the phone or
by meeting in person.

Another solution is to use \dfn{public-key} cryptography.  In a public
key system, there are two different keys: one for encryption and one for
decryption.  The encryption key can be made public by listing it in a
directory or mailing it to your correspondent, while you keep the
decryption key secret.  Your correspondent then sends you data encrypted
with your public key, and you use the private key to decrypt it.  While
the two keys are related, it's very difficult to derive the private key
given only the public key; however, deriving the private key is always
possible given enough time and computing power.  This makes it very
important to pick keys of the right size: large enough to be secure, but
small enough to be applied fairly quickly.

Many public-key algorithms can also be used to sign messages; simply
run the message to be signed through a decryption with your private
key key.  Anyone receiving the message can encrypt it with your
publicly available key and read the message.  Some algorithms do only
one thing, others can both encrypt and authenticate.

The currently available public-key algorithms are listed in the
following table:

\begin{tableii}{c|l}{}{Algorithm}{Capabilities}
\lineii{RSA}{Encryption, authentication/signatures}
\lineii{ElGamal}{Encryption, authentication/signatures}
\lineii{DSA}{Authentication/signatures}
\lineii{qNEW}{Authentication/signatures}
\end{tableii}

Many of these algorithms are patented.  Before using any of them in a
commercial product, consult a patent attorney; you may have to arrange
a license with the patent holder.

An example of using the RSA module to sign a message:
\begin{verbatim}
>>> from Crypto.Hash import MD5
>>> from Crypto.PublicKey import RSA
>>> RSAkey = RSA.generate(384, randfunc)   # This will take a while...
>>> hash = MD5.new(plaintext).digest()
>>> signature = RSAkey.sign(hash, "")
>>> signature   # Print what an RSA sig looks like--you don't really care.
('\021\317\313\336\264\315' ...,)
>>> RSAkey.verify(hash, signature)     # This sig will check out
1
>>> RSAkey.verify(hash[:-1], signature)# This sig will fail
0
\end{verbatim}

Public-key modules make the following functions available:

\begin{funcdesc}{construct}{tuple}
Constructs a key object from a tuple of data.  This is
algorithm-specific; look at the source code for the details.  (To be
documented later.)
\end{funcdesc}

\begin{funcdesc}{generate}{size, randfunc, progress_func=\code{None}}
Generate a fresh public/private key pair.  \var{size} is a
algorithm-dependent size parameter, usually measured in bits; the
larger it is, the more difficult it will be to break the key.  Safe
key sizes vary from algorithm to algorithm; you'll have to research
the question and decide on a suitable key size for your application.
An N-bit keys can encrypt messages up to N-1 bits long.

\var{randfunc} is a random number generation function; it should
accept a single integer \var{N} and return a string of random data
\var{N} bytes long.  You should always use a cryptographically secure
random number generator, such as the one defined in the
\module{Crypto.Util.randpool} module; \emph{don't} just use the
current time and the \module{random} module. 

\var{progress_func} is an optional function that will be called with a short
string containing the key parameter currently being generated; it's
useful for interactive applications where a user is waiting for a key
to be generated.
\end{funcdesc}

If you want to interface with some other program, you will have to know
the details of the algorithm being used; this isn't a big loss.  If you
don't care about working with non-Python software, simply use the
\module{pickle} module when you need to write a key or a signature to a
file.  It's portable across all the architectures that Python supports,
and it's simple to use.

Public-key objects always support the following methods.  Some of them
may raise exceptions if their functionality is not supported by the
algorithm.

\begin{methoddesc}{can_blind}{}
Returns true if the algorithm is capable of blinding data; 
returns false otherwise.  
\end{methoddesc}

\begin{methoddesc}{can_encrypt}{}
Returns true if the algorithm is capable of encrypting and decrypting
data; returns false otherwise.  To test if a given key object can sign
data, use \code{key.can_encrypt() and key.has_private()}.
\end{methoddesc}

\begin{methoddesc}{can_sign}{}
Returns true if the algorithm is capable of signing data; returns false
otherwise.  To test if a given key object can sign data, use
\code{key.can_sign() and key.has_private()}.
\end{methoddesc}

\begin{methoddesc}{decrypt}{tuple}
Decrypts \var{tuple} with the private key, returning another string.
This requires the private key to be present, and will raise an exception
if it isn't present.  It will also raise an exception if \var{string} is
too long.
\end{methoddesc}

\begin{methoddesc}{encrypt}{string, K}
Encrypts \var{string} with the private key, returning a tuple of
strings; the length of the tuple varies from algorithm to algorithm.  
\var{K} should be a string of random data that is as long as
possible.  Encryption does not require the private key to be present
inside the key object.  It will raise an exception if \var{string} is
too long.  For ElGamal objects, the value of \var{K} expressed as a
big-endian integer must be relatively prime to \code{self.p-1}; an
exception is raised if it is not.
\end{methoddesc}

\begin{methoddesc}{has_private}{}
Returns true if the key object contains the private key data, which
will allow decrypting data and generating signatures.
Otherwise this returns false.
\end{methoddesc}

\begin{methoddesc}{publickey}{}
Returns a new public key object that doesn't contain the private key
data. 
\end{methoddesc}

\begin{methoddesc}{sign}{string, K}
Sign \var{string}, returning a signature, which is just a tuple; in
theory the signature may be made up of any Python objects at all; in
practice they'll be either strings or numbers.  \var{K} should be a
string of random data that is as long as possible.  Different algorithms
will return tuples of different sizes.  \code{sign()} raises an
exception if \var{string} is too long.  For ElGamal objects, the value
of \var{K} expressed as a big-endian integer must be relatively prime to
\code{self.p-1}; an exception is raised if it is not.
\end{methoddesc}

\begin{methoddesc}{size}{}
Returns the maximum size of a string that can be encrypted or signed,
measured in bits.  String data is treated in big-endian format; the most
significant byte comes first.  (This seems to be a \emph{de facto} standard
for cryptographical software.)  If the size is not a multiple of 8, then
some of the high order bits of the first byte must be zero.  Usually
it's simplest to just divide the size by 8 and round down.
\end{methoddesc}

\begin{methoddesc}{verify}{string, signature}
Returns true if the signature is valid, and false otherwise.
\var{string} is not processed in any way; \code{verify} does
not run a hash function over the data, but you can easily do that yourself.
\end{methoddesc}

\subsection{The ElGamal and DSA algorithms}
For RSA, the \var{K} parameters are unused; if you like, you can just
pass empty strings.  The ElGamal and DSA algorithms require a real
\var{K} value for technical reasons; see Schneier's book for a detailed
explanation of the respective algorithms.  This presents a possible
hazard that can  
inadvertently reveal the private key.  Without going into the
mathematical details, the danger is as follows. \var{K} is never derived
or needed by others; theoretically, it can be thrown away once the
encryption or signing operation is performed.  However, revealing
\var{K} for a given message would enable others to derive the secret key
data; worse, reusing the same value of \var{K} for two different
messages would also enable someone to derive the secret key data.  An
adversary could intercept and store every message, and then try deriving
the secret key from each pair of messages.

This places implementors on the horns of a dilemma.  On the one hand,
you want to store the \var{K} values to avoid reusing one; on the other
hand, storing them means they could fall into the hands of an adversary.
One can randomly generate \var{K} values of a suitable length such as
128 or 144 bits, and then trust that the random number generator
probably won't produce a duplicate anytime soon.  This is an
implementation decision that depends on the desired level of security
and the expected usage lifetime of a private key.  I can't choose and
enforce one policy for this, so I've added the \var{K} parameter to the
\method{encrypt} and \method{sign} methods.  You must choose \var{K} by
generating a string of random data; for ElGamal, when interpreted as a
big-endian number (with the most significant byte being the first byte
of the string), \var{K} must be relatively prime to \code{self.p-1}; any
size will do, but brute force searches would probably start with small
primes, so it's probably good to choose fairly large numbers.  It might be
simplest to generate a prime number of a suitable length using the
\module{Crypto.Util.number} module.


\subsection{Security Notes for Public-key Algorithms}
Any of these algorithms can be trivially broken; for example, RSA can be
broken by factoring the modulus \emph{n} into its two prime factors.
This is easily done by the following code:

\begin{verbatim}
for i in range(2, n): 
    if (n%i)==0: 
        print i, 'is a factor' 
        break
\end{verbatim}

However, \emph{n} is usually a few hundred bits long, so this simple
program wouldn't find a solution before the universe comes to an end.
Smarter algorithms can factor numbers more quickly, but it's still
possible to choose keys so large that they can't be broken in a
reasonable amount of time.  For ElGamal and DSA, discrete logarithms are
used instead of factoring, but the principle is the same.

Safe key sizes depend on the current state of number theory and
computer technology.  At the moment, one can roughly define three
levels of security: low-security commercial, high-security commercial,
and military-grade.  For RSA, these three levels correspond roughly to
768, 1024, and 2048-bit keys.


%======================================================================
\section{Crypto.Util: Odds and Ends}
This chapter contains all the modules that don't fit into any of the
other chapters.  

\subsection{Crypto.Util.number}

This module contains various number-theoretic functions.  

\begin{funcdesc}{GCD}{x,y}
Return the greatest common divisor of \var{x} and \var{y}.
\end{funcdesc}

\begin{funcdesc}{getPrime}{N, randfunc}
Return an \var{N}-bit random prime number, using random data obtained
from the function \var{randfunc}.  \var{randfunc} must take a single
integer argument, and return a string of random data of the
corresponding length; the \method{get_bytes()} method of a
\class{RandomPool} object will serve the purpose nicely, as will the
\method{read()} method of an opened file such as \file{/dev/random}.
\end{funcdesc}

\begin{funcdesc}{getRandomNumber}{N, randfunc}
Return an \var{N}-bit random number, using random data obtained from the
function \var{randfunc}.  As usual, \var{randfunc} must take a single
integer argument and return a string of random data of the
corresponding length.
\end{funcdesc}

\begin{funcdesc}{inverse}{u, v}
Return the inverse of \var{u} modulo \var{v}.
\end{funcdesc}

\begin{funcdesc}{isPrime}{N}
Returns true if the number \var{N} is prime, as determined by a
Rabin-Miller test.
\end{funcdesc}


\subsection{Crypto.Util.randpool}

For cryptographic purposes, ordinary random number generators are
frequently insufficient, because if some of their output is known, it
is frequently possible to derive the generator's future (or past)
output.  Given the generator's state at some point in time, someone
could try to derive any keys generated using it.  The solution is to
use strong encryption or hashing algorithms to generate successive
data; this makes breaking the generator as difficult as breaking the
algorithms used.

Understanding the concept of \dfn{entropy} is important for using the
random number generator properly.  In the sense we'll be using it,
entropy measures the amount of randomness; the usual unit is in bits.
So, a single random bit has an entropy of 1 bit; a random byte has an
entropy of 8 bits.  Now consider a one-byte field in a database containing a
person's sex, represented as a single character \samp{M} or \samp{F}.
What's the entropy of this field?  Since there are only two possible
values, it's not 8 bits, but one; if you were trying to guess the value,
you wouldn't have to bother trying \samp{Q} or \samp{@}.  

Now imagine running that single byte field through a hash function that
produces 128 bits of output.  Is the entropy of the resulting hash value
128 bits?  No, it's still just 1 bit.  The entropy is a measure of how many
possible states of the data exist.  For English
text, the entropy of a five-character string is not 40 bits; it's
somewhat less, because not all combinations would be seen.  \samp{Guido}
is a possible string, as is \samp{In th}; \samp{zJwvb} is not.

The relevance to random number generation?  We want enough bits of
entropy to avoid making an attack on our generator possible.  An
example: One computer system had a mechanism which generated nonsense
passwords for its users.  This is a good idea, since it would prevent
people from choosing their own name or some other easily guessed string.
Unfortunately, the random number generator used only had 65536 states,
which meant only 65536 different passwords would ever be generated, and
it was easy to compute all the possible passwords and try them.  The
entropy of the random passwords was far too low.  By the same token, if
you generate an RSA key with only 32 bits of entropy available, there
are only about 4.2 billion keys you could have generated, and an
adversary could compute them all to find your private key.  See \rfc{1750},
"Randomness Recommendations for Security", for an interesting discussion
of the issues related to random number generation.

The \module{randpool} module implements a strong random number generator
in the \class{RandomPool} class.  The internal state consists of a string
of random data, which is returned as callers request it.  The class
keeps track of the number of bits of entropy left, and provides a function to
add new random data; this data can be obtained in various ways, such as
by using the variance in a user's keystroke timings.  

\begin{classdesc}{RandomPool}{\optional{numbytes, cipher, hash} }
An object of the \code{RandomPool} class can be created without
parameters if desired.  \var{numbytes} sets the number of bytes of
random data in the pool, and defaults to 160 (1280 bits). \var{hash}
can be a string containing the module name of the hash function to use
in stirring the random data, or a module object supporting the hashing
interface.  The default action is to use SHA.

The \var{cipher} argument is vestigial; it was removed from version
1.1 so RandomPool would work even in the limited exportable subset of
the code.  I recommend passing \var{hash} using a keyword argument so
that someday I can safely delete the \var{cipher} argument

\end{classdesc}

\class{RandomPool} objects define the following variables and methods:

\begin{methoddesc}{add_event}{time\optional{, string}}
Adds an event to the random pool.  \var{time} should be set to the
current system time, measured at the highest resolution available.
\var{string} can be a string of data that will be XORed into the pool,
and can be used to increase the entropy of the pool.  For example, if
you're encrypting a document, you might use the hash value of the
document; an adversary presumably won't have the plaintext of the
document, and thus won't be able to use this information to break the
generator.
\end{methoddesc}

The return value is the value of \member{self.entropy} after the data has
been added.  The function works in the following manner: the time
between successive calls to the \method{add_event()} method is determined,
and the entropy of the data is guessed; the larger the time between
calls, the better.  The system time is then read and added to the pool,
along with the \var{string} parameter, if present.  The hope is that the
low-order bits of the time are effectively random.  In an application,
it is recommended that \method{add_event()} be called as frequently as
possible, with whatever random data can be found.

\begin{memberdesc}{bits}
A constant integer value containing the number of bits of data in
the pool, equal to the \member{bytes} attribute multiplied by 8.
\end{memberdesc}

\begin{memberdesc}{bytes}
A constant integer value containing the number of bytes of data in
the pool.
\end{memberdesc}

\begin{memberdesc}{entropy}
An integer value containing the number of bits of entropy currently in
the pool.  The value is incremented by the \method{add_event()} method,
and decreased by the \method{get_bytes()} method.
\end{memberdesc}

\begin{methoddesc}{get_bytes}{num}
Returns a string containing \var{num} bytes of random data, and
decrements the amount of entropy available.  It is not an error to
reduce the entropy to zero, or to call this function when the entropy
is zero.  This simply means that, in theory, enough random information has been
extracted to derive the state of the generator.  It is the caller's
responsibility to monitor the amount of entropy remaining and decide
whether it is sufficent for secure operation.
\end{methoddesc}

\begin{methoddesc}{stir}{}
Scrambles the random pool using the previously chosen encryption and
hash function.  An adversary may attempt to learn or alter the state
of the pool in order to affect its future output; this function
destroys the existing state of the pool in a non-reversible way.  It
is recommended that \method{stir()} be called before and after using
the \class{RandomPool} object.  Even better, several calls to
\method{stir()} can be interleaved with calls to \method{add_event()}.
\end{methoddesc}

The \class{PersistentRandomPool} class is a subclass of \class{RandomPool} 
that adds the capability to save and load the pool from a disk file.

\begin{classdesc}{PersistentRandomPool}{filename, \optional{numbytes, cipher, hash}}
The path given in \var{filename} will be automatically opened, and an
existing random pool read; if no such file exists, the pool will be
initialized as usual.  If omitted, the filename defaults to the empty
string, which will prevent it from being saved to a file.  These
arguments are identical to those for the \class{RandomPool}
constructor.
\end{classdesc}

\begin{methoddesc}{save}{}
Opens the file named by the \member{filename} attribute, and saves the
random data into the file using the \module{pickle} module.
\end{methoddesc}

The \class{KeyboardRandomPool} class is a subclass of
\class{PersistentRandomPool} that provides a method to obtain random
data from the keyboard:

\begin{methoddesc}{randomize}{}
(Unix systems only)  Obtain random data from the keyboard.  This works
by prompting the
user to hit keys at random, and then using the keystroke timings (and
also the actual keys pressed) to add entropy to the pool.  This works
similarly to PGP's random pool mechanism.
\end{methoddesc}


\subsection{Crypto.Util.RFC1751}
The keys for private-key algorithms should be arbitrary binary data.
Many systems err by asking the user to enter a password, and then
using the password as the key.  This limits the space of possible
keys, as each key byte is constrained within the range of possible
ASCII characters, 32-127, instead of the whole 0-255 range possible
with ASCII.  Unfortunately, it's difficult for humans to remember 16
or 32 hex digits.

One solution is to request a lengthy passphrase from the user, and
then run it through a hash function such as SHA or MD5.  Another
solution is discussed in RFC 1751, "A Convention for Human-Readable
128-bit Keys", by Daniel L. McDonald.  Binary keys are transformed
into a list of short English words that should be easier to remember.
For example, the hex key EB33F77EE73D4053 is transformed to "TIDE ITCH
SLOW REIN RULE MOT".

\begin{funcdesc}{key_to_english}{key}
Accepts a string of arbitrary data \var{key}, and returns a string
containing uppercase English words separated by spaces.  \var{key}'s
length must be a multiple of 8.
\end{funcdesc}

\begin{funcdesc}{english_to_key}{string}
Accepts \var{string} containing English words, and returns a string of
binary data representing the key.  Words must be separated by
whitespace, and can be any mixture of uppercase and lowercase
characters.  6 words are required for 8 bytes of key data, so
the number of words in \var{string} must be a multiple of 6.
\end{funcdesc}


%======================================================================
\section{The Demonstration Programs}

The Python Cryptography Toolkit comes with various demonstration
programs, located in the \file{Demo/} directory.  None of them is
particularly well-finished or suitable for serious use.  Rather,
they're intended to illustrate how the toolkit is used, and to provide
some interesting possible uses.  Feel free to incorporate the code (or
modifications of it) into your own programs.

\subsection{Demo 1: \file{cipher}}

\file{cipher} encrypts and decrypts files.  On most Unix systems, the
\file{crypt} program uses a variant of the Enigma cipher.  This is not
secure, and there exists a freely available program called ``Crypt
Breaker's Workbench'' which helps in breaking the cipher if you have
some knowledge of the encrypted data.

\file{cipher} is a more secure file encryption program.  Simply list
the names of the files to be encrypted on the command line.
\file{cipher} will go through the list and encrypt or decrypt them;
\file{cipher} can recognize files it has previously encrypted.  The
ciphertext of a file is placed in a file of the same name with
'\samp{.cip}' appended; the original file is not deleted, since I'm
not sure that all errors during operation are caught, and I don't want
people to accidentally erase important files.

There are two command-line options: \programopt{-c} and
\programopt{-k}.  Both of them require an argument.  \code{-c
\var{ciphername}} uses the given encryption algorithm
\var{ciphername}; for example, \code{-c des} will use the DES
algorithm.  The name should be the same as an available module name;
thus it should be in lowercase letters.  The default cipher is IDEA.

\code{-k \var{key}} can be used to set the encryption key to be used.
Note that on a multiuser Unix system, the \code{ps} command can be
used to view the arguments of commands executed by other users, so
this is insecure; if you're the only user (say, on your home computer
running Linux) you don't have to worry about this.  If no key is set
on the command line, \file{cipher} will prompt the user to input a key
on standard input.


\subsubsection{Technical Details}

The encrypted file is not pure ciphertext.  First comes a magic
string; this is currently the sequence \samp{ctx} and a byte
containing 1 (the version number of \file{cipher}).
This is followed by the null-terminated name of the encryption
algorithm, and the rest of the file contains the ciphertext.  

The plaintext is encrypted in CBC mode.  The initial value for the
feedback is always set to a block filled with the letter 'A', and then
a block of random data is encrypted.  This garbage block will be
discarded on decryption.  Note that the random data is not generated
in a cryptographically secure way, and this may provide a tiny foothold for
an attacker.

After the random block is generated, the magic string, length of the
original file, and original filename are all encrypted before the file
data is finally processed.  Some extra characters of padding may be
added to obtain an integer number of blocks.  This padding will also
be discarded on decryption.  Note that the plaintext file will be
completely read into memory before encryption is performed; no
buffering is done.  Therefore, don't encrypt 20-megabyte files unless
you're willing to face the consequences of a 20-megabyte process.

Areas for improvements to \file{cipher} are: cryptographically secure
generation of random data for padding, key entry, and buffering of
file input.


\subsection{Demo 2: \file{secimp} and \file{sign}}

\file{secimp} demonstrates an application of the Toolkit that may be
useful if Python is being used as an extension language for mail and Web
clients: secure importing of Python modules.  To use it, run
\file{sign.py} in a directory with several compiled Python files
present.  It will use the key in \file{testkey.py} to generate digital
signatures for the compiled Python code, and save both the signature and
the code in a file ending in \samp{.pys}.  Then run \code{python -i
secimp.py}, and import a file by using \code{secimport}.  

For example, if \file{foo.pys} was constructed, do
\code{secimport('foo')}.  The import should succeed.  Now fire up Emacs
or some other editor, and change a string in the code in \file{foo.pys};
you might try changing a letter in the name of a variable.  When you run
\code{secimport('foo')}, it should raise an exception reporting the
failed signature.  If you execute the statement \code{__import__ =
secimport}, the secure import will be used by default for all future
module imports.  Alternatively, if you were creating a restricted
execution environment using \file{rexec.py}, you could place
\code{secimport()} in the restricted environment's namespace as the
default import function.


%======================================================================
\section{Extending the Toolkit}

Preserving the a common interface for cryptographic routines is a good
idea.  This chapter explains how to write new modules for the Toolkit.

The basic process is as follows:
\begin{enumerate}

\item Add a new \file{.c} file containing an implementation of the new
algorithm.  
This file must define 3 or 4 standard functions,
a few constants, and a C \code{struct} encapsulating the state variables required by the algorithm.

\item  Add the new algorithm to \file{setup.py}.

\item  Send a copy of the code to me, if you like; code for new
algorithms will be gratefully accepted.
\end{enumerate}


\subsection{Adding Hash Algorithms}

The required constant definitions are as follows:

\begin{verbatim}
#define MODULE_NAME MD2		/* Name of algorithm */
#define DIGEST_SIZE 16          /* Size of resulting digest in bytes */
\end{verbatim}

The C structure must be named \ctype{hash_state}:

\begin{verbatim}
typedef struct {
     ... whatever state variables you need ...
} hash_state;
\end{verbatim}

There are four functions that need to be written: to initialize the
algorithm's state, to hash a string into the algorithm's state, to get
a digest from the current state, and to copy a state.

\begin{itemize}
  \item \code{void hash_init(hash_state *self);}
  \item \code{void hash_update(hash_state *self, unsigned char *buffer, int length);}
  \item \code{PyObject *hash_digest(hash_state *self);}
  \item \code{void hash_copy(hash_state *source, hash_state *dest);}
\end{itemize}

Put \code{\#include "hash_template.c"} at the end of the file to
include the actual implementation of the module.


\subsection{Adding Block Encryption Algorithms}

The required constant definitions are as follows:

\begin{verbatim}
#define MODULE_NAME AES	       /* Name of algorithm */
#define BLOCK_SIZE 16          /* Size of encryption block */
#define KEY_SIZE 0             /* Size of key in bytes (0 if not fixed size) */
\end{verbatim}

The C structure must be named \ctype{block_state}:

\begin{verbatim}
typedef struct {
     ... whatever state variables you need ...
} block_state;
\end{verbatim}

There are three functions that need to be written: to initialize the
algorithm's state, and to encrypt and decrypt a single block.

\begin{itemize}
  \item \code{void block_init(block_state *self, unsigned char *key,
                int keylen);}
  \item \code{void block_encrypt(block_state *self, unsigned char *in, 
               unsigned char *out);}
  \item \code{void block_decrypt(block_state *self, unsigned char *in, 
               unsigned char *out);}
\end{itemize}

Put \code{\#include "block_template.c"} at the end of the file to
include the actual implementation of the module.


\subsection{Adding Stream Encryption Algorithms}

The required constant definitions are as follows:

\begin{verbatim}
#define MODULE_NAME ARC4       /* Name of algorithm */
#define BLOCK_SIZE 1           /* Will always be 1 for a stream cipher */
#define KEY_SIZE 0             /* Size of key in bytes (0 if not fixed size) */
\end{verbatim}

The C structure must be named \ctype{stream_state}:

\begin{verbatim}
typedef struct {
     ... whatever state variables you need ...
} stream_state;
\end{verbatim}

There are three functions that need to be written: to initialize the
algorithm's state, and to encrypt and decrypt a single block.

\begin{itemize}
  \item \code{void stream_init(stream_state *self, unsigned char *key,
                int keylen);}
  \item \code{void stream_encrypt(stream_state *self, unsigned char *block, 
               int length);}
  \item \code{void stream_decrypt(stream_state *self, unsigned char *block, 
               int length);}
\end{itemize}

Put \code{\#include "stream_template.c"} at the end of the file to
include the actual implementation of the module.


\end{document}
