\documentclass{howto}

\title{The Python Cryptography Modules}

\release{1.1.0}

\author{A.M. Kuchling}
\authoraddress{\email{akuchling@acm.org}}

\begin{document}
\maketitle

\begin{abstract}
\noindent
This document describes a package containing various cryptographic
modules available for the Python programming language.  It assumes you
have some basic knowledge about the Python language and about
cryptography in general.
\end{abstract}

\tableofcontents

\section{Introduction}

\subsection{Design Goals}
The Python cryptography modules are intended to provide a reliable and
stable base for writing Python programs that require
cryptographic functions.  

A central goal of the author's has been to provide a
simple, consistent interface for similar classes of algorithms.  For
example, all block cipher objects have the same methods and return
values, and support the same feedback modes; hash functions have a
different interface, but it too is consistent over all the
hash functions available.  Individual modules also define variables to
help you write Python code that doesn't depend on the algorithms used;
for example, each block cipher module defines a variable that gives
the algorithm's block size.  This is intended to make it easy to
replace old algorithms with newer, more secure ones.  If you're given
a bit of portably-written Python code that uses the DES encryption
algorithm, you should be able to use IDEA instead by simply changing
\code{from Crypto.Cipher import DES} to \code{from Crypto.Cipher import
idea}, and changing all references to  
\code{DES.new()} to \code{IDEA.new()}.  It's also fairly simple to write
your own modules that mimic this interface, thus letting you use
combinations or permutations of algorithms.

\index{C language}
\index{language, C}
\index{Intel}
Some modules are implemented in C for performance; others are written in
Python for ease of modification.  Generally, low-level functions like
ciphers and hash functions are written in C, while less speed-critical
functions have been written in Python.  This division may change in
future releases.  When speeds are quoted in this document, they were
measured on a 266 MHz Pentium II running Linux.  The exact speeds will obviously
vary with different machines and different compilers, but they provide a
basis for comparing algorithms.  Currently the cryptographic
implementations are acceptably fast, but not spectacularly good.  I
welcome any suggestions or patches for faster code.  

% You may be surprised that I'm distributing the code worldwide.  Aren't
% North Americans prohibited from exporting cryptographic software?  Well,
% yes and no.  American citizens have to get permission from the US
% government before exporting \emph{any} such software, even if it's
% publicly available or originated outside the US; there are no exceptions
% whatsoever.

% The laws in Canada are slightly less restrictive, though, and I'm a
% Canadian citizen.  In particular, free software is specifically exempted
% from export restrictions, so I'm acting within the law.  However, if
% you're a US citizen, you \emph{still} can't export these programs
% without government permission.

\index{ITAR, regulations}
\index{regulations, ITAR}
If you live outside of Canada or the US, please do not attempt to
download it from a North American FTP site; you may get the site's
maintainer in trouble.  Documentation is not covered by the ITAR
regulations, and can be freely sent anywhere in the world.

\index{licensing terms}
I have placed the code under no restrictions; you can redistribute the
code freely or commercially, in its original form or with any
modifications you make, subject to whatever local laws may apply in your
jurisdiction.  Note that you still have to come to some agreement with
the holders of any patented algorithms you're using.  If you're
intensively using these modules, please tell me about it; there's little
incentive for me to work on this package if I don't know of anyone using
it.

I also make no guarantees as to the usefulness, correctness, or legality
of these modules, nor does their inclusion constitute an endorsement of
their effectiveness.  Many cryptographic algorithms are patented;
inclusion in this package does not necessarily mean you are allowed to
incorporate them in a product and sell it.  Some of these algorithms may
have been cryptanalyzed, and may no longer be secure.  While I will
include commentary on the relative security of the algorithms in the
sections entitled "Security Notes", there may be more recent analyses
I'm not aware of.  (Or maybe I'm just clueless.)  If you're implementing
an important system, don't just grab things out of a toolbox and put
them together; do some research first.  On the other hand, if you're
just interested in keeping your co-workers or your relatives out of your
files, any of the components here could be used.

This document is very much a work in progress.  If you have any
questions, comments, complaints, or suggestions, please send them to me
at \email{akuchling@acm.org}.

\subsection{Acknowledgements}
\index{Schneier, Bruce}
Much of the code that actually implements the various cryptographic
algorithms was not written by me.  I'd like to thank all the people who
implemented them, and released their work under terms which allowed me
to use their code.  The individuals are credited in the relevant
chapters of this documentation.  Bruce Schneier's book \emph{Applied
Cryptography} was also very useful in writing this toolkit; I highly
recommend it if you're interested in learning more about cryptography.
Mr. Schneier also has a Web site at \url{http://www.counterpane.com}.

Good luck with your cryptography hacking!

A.M.K.

\email{akuchling@acm.org}

Washington, DC, USA

January 1998

\section{Crypto.Hash: Hash Functions}

\index{MD2 (hash function)}
\index{MD5 (hash function)}
\index{SHA (hash function)}
\index{HAVAL (hash function)}
Hash functions take arbitrary strings as input, and produce an output
of fixed size that is dependent on the input; it should never be
possible to derive the input data given only the hash function's
output.  One simple hash function consists of simply adding together
all the bytes of the input, and taking the result modulo 256.  For a
hash function to be cryptographically secure, it must be very
difficult to find two messages with the same hash value, or to find a
message with a given hash value.  The simple additive hash function
fails this criterion miserably; the hash functions described below do
not.  Examples of cryptographically secure hash functions include MD2,
MD5, SHA, and HAVAL.

Hash functions can be used simply as a checksum, or, in association with a
public-key algorithm, can be used to implement digital signatures.
 
The hashing algorithms currently implemented are listed in the following table:

\begin{tableii}{c|l}{}{Hash function}{Digest length}
\lineii{HAVAL}{Variable size: 128, 160, 192, 224, or 256 bits}
\lineii{MD2}{128 bits}
\lineii{MD4}{128 bits}
\lineii{MD5}{128 bits}
\lineii{RIPEMD}{160 bits}
\lineii{SHA}{160 bits}
\end{tableii}

All hashing modules share the same interface.  After importing a given
hashing module, call the \code{new()} function to create a new hashing
object. (In older versions of the Python interpreter, the \code{md5()}
function was used to perform this task.  If you're modifying an old script, you
should change any calls to \code{md5()} to use \code{new()} instead.)  You
can now feed arbitrary strings into the object, and can ask for the
hash value at any time.  The \code{new()} function can also be passed an
optional string parameter, which will be hashed immediately.

Hash function modules define one variable:

\begin{datadesc}{digestsize}
An integer value; the size of the digest
produced by the hashing objects.  You could also obtain this value by
creating a sample object, and taking the length of the digest string
it returns, but using \code{digestsize} is faster.
\end{datadesc}

The methods for hashing objects are always the following:

\begin{funcdesc}{copy}{}
Return a separate copy of this hashing object.  An \code{update} to this
  copy won't affect the original object.
\end{funcdesc}

\begin{funcdesc}{digest}{}
Return the hash value of this hashing object, as a string containing 8-bit data.  The object is not
altered in any way by this function; you can continue updating the
object after calling this function.
\end{funcdesc}

\begin{funcdesc}{digest}{}
Return the hash value of this hashing object, as a string containing the
digest data as hexadecimal digits.  The resulting string will be twice
as long as that returned by \code{digest()}.  The object is not altered
in any way by this function; you can continue updating the object after
calling this function.
\end{funcdesc}

\begin{funcdesc}{update}{arg}
Update this hashing object with the string \var{arg}.
\end{funcdesc}

\index{RSA Data Security, Inc.}
\index{MD5 (hash function)}
Here's an example, using RSA Data Security's MD5 algorithm:

\begin{verbatim}
>>> from Crypto.Hash import MD5
>>> m = MD5.new()
>>> m.update('abc')
>>> m.digest()
'\220\001P\230<\322O\260\326\226?@}(\341\177r'
\end{verbatim}


Or, more compactly:
\begin{verbatim}
>>> MD5.new('abc').digest()
'\220\001P\230<\322O\260\326\226?@}(\341\177r'
\end{verbatim}

\subsection{Algorithm-specific Notes for Hash Functions}

HAVAL provides a variable-size digest, and allows for a variable number
of rounds.  It's believed that increasing the number of rounds increases
the security; at least, I don't know of any results to the contrary.
The \code{HAVAL.new()} accordingly has two keyword arguments,
\code{rounds} and \code{digestsize}.  \code{rounds} can be 3, 4, or 5,
and has a default value of 5.  \code{digestsize} can be 128, 160, 192,
224, or 256 bits, and has a default value of 256. 

\subsection{Security Notes}
Hashing algorithms are broken when it's easy to compute a
string that produces a given hash value, or to find two
messages that produce the same hash value. Consider an example where
Alice and Bob are using digital signatures to sign a contract.  Alice
computes the hash value of the text of the contract and signs it.  Bob
could then compute a different contract that has the same hash value,
and it would appear that Alice has signed that bogus contract; she'd
have no way to prove otherwise.  Finding such a message by brute force
takes \code{pow(2, b-1)} operations, where the hash function produces
\emph{b}-bit hashes.

If Bob can only find two messages with the same hash value but can't
choose the resulting hash value, he can look for two messages with
different meanings, such as "I will mow Bob's lawn for $10" and "I owe
Bob $1,000,000", and ask Alice to sign the first, innocuous contract.
This attack is easier for Bob, since finding two such messages by brute
force will take \code{pow(2, b/2)} operations on average.  However,
Alice can protect herself by changing the protocol; she can simply
append a random string to the contract before hashing and signing it;
the random string can then be kept with the signature.

None of the algorithms implemented here have been completely broken.
There are no attacks on MD2, but it's rather slow at 376 K/sec.  MD4 is
faster at 11946 K/sec but there have been some partial attacks on it.  MD4
operates in three iterations of a basic mixing operation; two of the
three rounds have been cryptanalyzed, but the attack can't be extended
to the full algorithm.  MD5 is a strengthened version of MD4 with four
rounds; an attack against one round has been found.  XXX Dobbertin's
attack.  Because MD5 is more
commonly used, the implementation is better optimized and thus faster on
x86 processors (27193 K/sec).  MD4 may be faster than MD5 when other
processors and compilers are used.

All the MD algorithms produce 128-bit hashes; SHA produces a larger 160-bit
hash, and there are no known attacks against it.  The first version of
SHA had a weakness which was later corrected; the code used here
implements the second, corrected, version.  It operates at 13157 K/sec.
RIPEMD also has a 160-bit output, and operates at 5869 K/sec.
HAVAL is a variable-size hash function; it can generate hash values that
are 128, 160, 192, 224, or 256 bits in size, and can use 3, 4, or 5
rounds.  5-round HAVAL runs at 5371 K/sec.

\subsection{Credits}
\index{Plumb, Colin}
\index{Kuchling, Andrew}
\index{Gutmann, Peter}
The MD2, MD4, and HAVAL implementations were written by A.M. Kuchling,
and the MD5 code was implemented by Colin Plumb.  The SHA code was
originally written by Peter Gutmann.  The RIPEMD code was written by
Antoon Bosselaers, and adapted for the toolkit by Hirendra Hindocha.

\section{Crypto.Cipher: Encryption Algorithms}
Encryption algorithms transform their input data (called
\dfn{plaintext}) in some way that is
dependent on a variable \dfn{key}, producing \dfn{ciphertext};
this transformation can easily be reversed, if (and, hopefully, only
if) one knows the key.  The key can be varied by the user or
application, chosen from some very large space of possible keys. 

For a secure encryption algorithm, it should be very difficult to
determine the original plaintext without knowing the key; usually, no
clever attacks on the algorithm are known, so the only way of breaking
the algorithm is to try all possible keys. Since the number of possible
keys is usually of the order of 2 to the power of 56 or 128, this is not
a serious threat, although 2 to the power of 56 is now considered
insecure in the face of custom-built parallel computers and distributed
key guessing efforts.

\index{feedback mode, ECB}
\dfn{Block ciphers} take multibyte inputs of a fixed size
(frequently 8 or 16 bytes long) and encrypt them.  Block ciphers can
be operated in various modes.  The simplest is Electronic Code Book
(or ECB) mode.  In this mode, each block of plaintext is simply
encrypted to produce the ciphertext.  This mode can be dangerous,
because many files will contain patterns greater than the block size;
for example, the comments in a C program may contain long strings of
asterisks intended to form a box.  All these identical blocks will
encrypt to identical ciphertext; an adversary may be able to use this
structure to obtain some information about the text.

\index{feedback mode, CBC}
\index{feedback mode, CFB}
To eliminate this weakness, there are various feedback modes, where
the plaintext is combined with the previous ciphertext before
encrypting; this eliminates any such structure.  One mode is Cipher
Block Chaining (CBC mode); another is Cipher FeedBack (CFB
mode).   CBC mode still encrypts in blocks, and thus is only
slightly slower than ECB mode.  CFB mode encrypts on a byte-by-byte
basis, and is much slower than either of the other two modes.  The
chaining feedback modes require an initialization value to start off
the encryption; this is a string of the same length as the ciphering
algorithm's block size, and is passed to the \code{new()} function.

There is also a special PGP mode, which is a variant
of CFB used by the PGP program.  While you can use it in non-PGP
programs, it's quite non-standard.

The currently available block ciphers are listed in the following table,
and are available in the \code{Crypto.Cipher} package:

\index{DES (block cipher)}
\index{DES3 (block cipher)}
\index{Triple DES (block cipher)}
\index{Diamond (block cipher)}
\index{RC5 (block cipher)}
\index{IDEA (block cipher)}
\index{Blowfish (block cipher)}
\index{ARC2 (block cipher)}
\index{CAST (block cipher)}

\begin{tableii}{c|l}{}{Cipher}{Key Size/Block Size}
\lineii{ARC2}{Variable/8 bytes}
\lineii{Blowfish}{Variable/8 bytes}
\lineii{CAST}{Variable/8 bytes}
\lineii{DES}{8 bytes/8 bytes}
\lineii{DES3 (Triple DES)}{16 bytes/8 bytes}
\lineii{Diamond}{Variable/16 bytes}
\lineii{IDEA}{16 bytes/8 bytes}
\lineii{RC5}{Variable/8 bytes}
\end{tableii}

\index{stream cipher}
In a strict formal sense, \dfn{stream ciphers} encrypt data bit-by-bit;
practically, stream ciphers work on a character-by-character basis.
Stream ciphers use exactly the
same interface as block ciphers, with a block length that will always
be 1; this is how block and stream ciphers can be distinguished. 
The only feedback mode available for stream ciphers is ECB mode. 

The currently available stream ciphers are listed in the following table:

\index{Sapphire (stream cipher)}
\index{ARC4 (stream cipher)}
\begin{tableii}{c|l}{}{Cipher}{Key Size}
\lineii{Cipher}{Key Size}
\lineii{ARC4}{Variable}
\lineii{Sapphire}{Variable}
\end{tableii}

\index{RSA Data Security, Inc.}
\index{sci.crypt}
\index{RC4 (stream cipher)}
\index{ARC4 (stream cipher)}
ARC4 is short for `Alleged RC4'.  The real RC4 algorithm is proprietary
to RSA Data Security Inc.  In September of 1994, someone posted C code
to both the Cypherpunks mailing list and to the Usenet newsgroup
\code{sci.crypt}, claiming that it implemented the RC4 algorithm.  This
posted code is what I'm calling Alleged RC4, or ARC4 for short.  I don't
know if ARC4 is in fact RC4, but ARC4 has been subjected to scrutiny on
the Cypherpunks mailing list and elsewhere, and does not seem to be
easily breakable.  The legal issues surrounding the use of ARC4 are
unclear, but be aware that it hasn't been subject to much scrutiny, and
may have some critical flaw that hasn't yet been discovered.  The same
is true of ARC2, which was posted in January, 1996.

An example usage of the DES module:
\begin{verbatim}
>>> from Crypto.Cipher import DES
>>> obj=DES.new('abcdefgh', DES.ECB)
>>> plain="Guido van Rossum is a space alien."
>>> len(plain)
34
>>> obj.encrypt(plain)
Traceback (innermost last):
  File "<stdin>", line 1, in ?
ValueError: Strings for DES must be a multiple of 8 in length
>>> ciph=obj.encrypt(plain+'XXXXXX')
>>> ciph
'\021,\343Nq\214DY\337T\342pA\372\255\311s\210\363,\300j\330\250\312\347\342I\3215w\03561\303dgb/\006'
>>> obj.decrypt(ciph)
'Guido van Rossum is a space alien.XXXXXX'
\end{verbatim}


All cipher algorithms share a common interface.  After importing a
given module, there is exactly one function and two variables
available.

\begin{funcdesc}{new}{key, mode\optional{, IV}}
Returns a ciphering object, using \var{key} and feedback mode
\var{mode}.  If \var{mode} is CBC or CFB, \var{IV} must be provided,
and must be a string of the same length as the block size.  Some
algorithms support additional keyword arguments to this function; see
the "Algorithm-specific Notes for Encryption Algorithms" section below for the details.
\end{funcdesc}

\begin{datadesc}{blocksize}
An integer value; the size of the blocks encrypted by this module.
Strings passed to the \code{encrypt} and \code{decrypt} functions
must be a multiple of this length.  For stream ciphers,
\code{blocksize} will be 1. 
\end{datadesc}

\begin{datadesc}{keysize}
An integer value; the size of the keys required by this module.  If
\code{keysize} is zero, then the algorithm accepts arbitrary-length
keys.  You cannot pass a key of length 0 (that is, the null string
\code{''} as such a variable-length key.  
\end{datadesc}

All cipher objects have at least three attributes:

\begin{datadesc}{blocksize}
An integer value equal to the size of the blocks encrypted by this object.
Identical to the module variable of the same name.
\end{datadesc}

\begin{datadesc}{IV}
Contains the initial value which will be used to start a cipher
feedback mode.  After encrypting or decrypting a string, this value
will reflect the modified feedback text; it will always be one block
in length.  It is read-only, and cannot be assigned a new value.
\end{datadesc}

\begin{datadesc}{keysize}
An integer value equal to the size of the keys used by this object.  If
\code{keysize} is zero, then the algorithm accepts arbitrary-length
keys.  For algorithms that support variable length keys, this will be 0.
Identical to the module variable of the same name.  
\end{datadesc}

All ciphering objects have the following methods:

\begin{funcdesc}{decrypt}{string}
Decrypts \var{string}, using the key-dependent data in the object, and
with the appropriate feedback mode.  The string's length must be an exact
multiple of the algorithm's block size.  Returns a string containing
the plaintext.
\end{funcdesc}

\begin{funcdesc}{encrypt}{string}
Encrypts a non-null \var{string}, using the key-dependent data in the
object, and with the appropriate feedback mode.  The string's length
must be an exact multiple of the algorithm's block size; for stream
ciphers, the string can be of any length.  Returns a string containing
the ciphertext.
\end{funcdesc}

\subsection{Algorithm-specific Notes for Encryption Algorithms}

The Diamond block cipher allows you to select the number of rounds to
apply, ranging from 5 to 15 (inclusive.)  This is set via the
\code{rounds} keyword argument to the \code{new()} function; the default
value is 8 rounds.

RC5 has even more parameters; see Ronald Rivest's paper at \url{http://theory.lcs.mit.edu/~rivest/rc5rev.ps}
for the implementation details.  The keyword parameters are:

\begin{itemize}
\item \code{version}:
The version
of the RC5 algorithm to use; currently the only legal value is
\code{0x10} for RC5 1.0.  
\item \code{wordsize}:
The word size to use;
16 or 32 are the only legal values.  (A larger word size is better, so
usually 32 will be used.  16-bit RC5 is probably only of academic
interest.)  
\item \code{rounds}:
The number of rounds to apply, the larger the more secure: this
can be any value from 0 to 255, so you will have to choose a value
balanced between speed and security. 
\end{itemize}

\subsection{Security Notes}
Encryption algorithms can be broken in several ways.  If you have some
ciphertext and know (or can guess) the corresponding plaintext, you can
simply try every possible key in a \dfn{known-plaintext} attack.  Or, it
might be possible to encrypt text of your choice using an unknown key;
for example, you might mail someone a message intending it to be
encrypted and forwarded to someone else.  This is a
\dfn{chosen-plaintext} attack, which is particularly effective if it's
possible to choose plaintexts that reveal something about the key when
encrypted.

DES (2458 K/sec) has a 56-bit key; this is starting to become too small
for safety.  It has been estimated that it would only cost \$1,000,000 to
build a custom DES-cracking machine that could find a key in 3 hours.  A
chosen-ciphertext attack using the technique of \dfn{linear
cryptanalysis} can break DES in \code{pow(2, 43)} steps.  However,
unless you're encrypting data that you want to be safe from major
governments, DES will be fine. DES3 (509 K/sec) uses three DES
encryptions for greater security and a 112-bit or 168-bit key, but is
correspondingly slower.

There are no publicly known attacks against IDEA (2005 K/sec), and
it's been around long enough to have been examined.  There are no
known attacks against ARC2 (1644 K/sec), ARC4 (4630 K/sec), Blowfish
(5323 K/sec), CAST (1314 K/sec), Diamond (1670 K/sec), RC5 (1370
K/sec), or Sapphire (3568 K/sec), but they're all relatively new
algorithms and there hasn't been time for much analysis to be
performed; use them for serious applications only after careful
research.  Skipjack (864 K/sec) is relatively slow, and its 80-bit key size
is secure at the moment, but doesn't leave much of a margin of safety;
only use it if you believe that the NSA has mystical cipher design
abilities.


\subsection{Credits}
\index{Olson, Bryan}
\index{Schneier, Bruce}
\index{Young, Eric}
\index{Cyphers, Graven}
\index{Outerbridge, Richard}
\index{Brown, Lawrence}
\index{Kwan, Matthew}
\index{DES (block cipher)}
\index{IDEA (block cipher)}
\index{Blowfish (block cipher)}
The code for Blowfish was written by Bryan Olson, partially based on a
previous implementation by Bruce Schneier, who also invented the
algorithm; the Blowfish algorithm has been placed in the public domain
and can be used freely.  (See \url{http://www.counterpane.com} for more
information about Blowfish.)  The CAST implementation was written by 
Wim Lewis.  The DES implementation was written by Eric Young, and the
IDEA implementation by Colin Plumb. The RC5 implementation
was written by A.M. Kuchling.

\index{sci.crypt}
\index{Johnson, Michael Paul}
\index{ARC4 (stream cipher)}
\index{Sapphire (stream cipher)}
The Alleged RC4 code was posted to the \code{sci.crypt} newsgroup by an
unknown party, and re-implemented by A.M. Kuchling.  The Sapphire stream
cipher was developed by Michael P. Johnson, and is in the public domain;
the implementation used here was written by A.M. Kuchling and is based
on Johnson's code.

\section{Crypto.Protocol: Various Protocols}

\subsection{Crypto.Protocol.AllOrNothing}

This module implements all-or-nothing package transformations.
An all-or-nothing package transformation is one in which some text is
transformed into message blocks, such that all blocks must be obtained before
the reverse transformation can be applied.  Thus, if any blocks are corrupted
or lost, the original message cannot be reproduced.

An all-or-nothing package transformation is not encryption, although a block
cipher algorithm is used.  The encryption key is randomly generated and is
extractable from the message blocks.

This class implements the All-Or-Nothing package transformation
algorithm described in Rivest: ``All-Or-Nothing Encryption and The
Package Transform.''  To appear in the Proceedings of the 1997 Fast
Software Encryption Conference.
http://theory.lcs.mit.edu/~rivest/fusion.ps

\begin{classdesc}{AllOrNothing}{ciphermodule, mode=None, IV=None}
Class implementing the All-or-Nothing package transform.

\var{ciphermodule} is a module implementing the cipher algorithm to
use.  Optional arguments \var{mode} and \var{IV} are passed directly
through to the \var{ciphermodule}.\code{new()} method; they are the
feedback mode and initialization vector to use.  All three arguments
must be the same for the object used to create the digest, and to
undigest'ify the message blocks.

The module passed as \var{ciphermodule} must provide the
following interface:

\var{ciphermodule}.\code{keysize}: 
Attribute containing the cipher algorithm's key size in
bytes.  If the cipher supports variable length keys, then
typically \code{ciphermodule.keysize} will be zero.  In that case a
key size of 16 bytes will be used.

\var{ciphermodule}.\code{blocksize}: 
Attribute containing the cipher algorithm's input block size
in bytes.

\var{ciphermodule}.\code{new}(\var{key}, \var{mode}, \var{IV}):
                Function which returns a new instance of a cipher object,
                initialized to \var{key}.  The returned object must have an
                \method{encrypt()} method that accepts a string of
                \var{ciphermodule}.\code{blocksize} bytes and returns a string containing
                the encrypted text.

Note that the encryption key is randomly generated automatically
when needed.  
\end{classdesc}

The methods of the \class{AllorNothing} class are:

\begin{methoddesc}{digest}{}
Perform the All-or-Nothing package transform on the current
string.  Output is a list of message blocks describing the
transformed text, where each block is a string of bit length equal
to the cipher module's blocksize.
\end{methoddesc}

\begin{methoddesc}{reset}{text = ""}
Reset the current string to be transformed to \var{text}.
\end{methoddesc}

\begin{methoddesc}{undigest}{mblocks}
Perform the reverse package transformation on a list of message
blocks.  Note that the cipher module used for both transformations
must be the same.  \var{mblocks} is a list of strings of bit length
equal to \var{ciphermodule}'s blocksize.  The output is a string object.
\end{methoddesc}

\begin{methoddesc}{update}{text}
Concatenate \var{text} to the string that will be transformed.
\end{methoddesc}

\subsection{Crypto.Protocol.Chaffing}

Winnowing and chaffing is a technique for enhancing privacy without requiring
strong encryption.  In short, the technique takes a set of authenticated
message blocks (the wheat) and adds a number of chaff blocks which have
randomly chosen data and MAC fields.  This means that to an adversary, the
chaff blocks look as valid as the wheat blocks, and so the authentication
would have to be performed on every block.  By tailoring the number of chaff
blocks added to the message, the sender can make breaking the message
computationally infeasible.  There are many other interesting properties of
the winnow/chaff technique.

For example, say Alice is sending a message to Bob.  She packetizes the
message and performs an all-or-nothing transformation on the packets.  Then
she authenticates each packet with a message authentication code (MAC).  The
MAC is a hash of the data packet, and there is a secret key which she must
share with Bob (key distribution is an exercise left to the reader).  She then
adds a serial number to each packet, and sends the packets to Bob.

Bob receives the packets, and using the shared secret authentication key,
authenticates the MACs for each packet.  Those packets that have bad MACs are
simply discarded.  The remainder are sorted by serial number, and passed
through the reverse all-or-nothing transform.  The transform means that an
eavesdropper (say Eve) must acquire all the packets before any of the data can
be read.  If even one packet is missing, the data is useless.

There's one twist: by adding chaff packets, Alice and Bob can make Eve's job
much harder, since Eve now has to break the shared secret key, or try every
combination of wheat and chaff packet to read any of the message.  The cool
thing is that Bob doesn't need to add any additional code; the chaff packets
are already filtered out because their MACs don't match (in all likelihood --
since the data and MACs for the chaff packets are randomly chosen it is
possible, but very unlikely that a chaff MAC will match the chaff data).  And
Alice need not even be the party adding the chaff!  She could be completely
unaware that a third party, say Charles, is adding chaff packets to her
messages as they are transmitted.

For more information on winnowing and chaffing see this paper:

XXX Rivest.

\begin{classdesc}{Chaff}{factor=1.0, blocksper=1}
Class implementing the chaff adding algorithm. 
\var{factor} is the number of message blocks 
            to add chaff to, expressed as a percentage between 0.0 and 1.0; the default value is 1.0.
\var{blocksper} is the number of chaff blocks to include for each block
            being chaffed, and defaults to 1.  The default settings 
add one chaff block to every
            message block.  By changing the defaults, you can adjust how
            computationally difficult it could be for an adversary to
            brute-force crack the message.  The difficulty is expressed as:

\begin{verbatim}
pow(blocksper, int(factor * number-of-blocks))
\end{verbatim}

For ease of implementation, when \var{factor} < 1.0, only the first
\code{int(\var{factor}*number-of-blocks)} message blocks are chaffed.
\end{classdesc}

\class{Chaff} instances have the following methods:

\begin{methoddesc}{chaff}{blocks}
Add chaff to message blocks.  \var{blocks} is a list of 3-tuples of the
form (\var{serial-number}, \var{data}, \var{MAC}).

Chaff is created by choosing a random number of the same
byte-length as \var{data}, and another random number of the same
byte-length as \var{MAC}.  The message block's serial number is placed
on the chaff block and all the packet's chaff blocks are randomly
interspersed with the single wheat block.  This method then
returns a list of 3-tuples of the same form.  Chaffed blocks will
contain multiple instances of 3-tuples with the same serial
number, but the only way to figure out which blocks are wheat and
which are chaff is to perform the MAC hash and compare values.
\end{methoddesc}

        Subclass methods:

\begin{methoddesc}{__randnum}{size}
Returns a randomly generated number with a byte-length equal
to \var{size}.  Subclasses can use this to implement better random
data and MAC generating algorithms.  The default algorithm is
probably not very cryptographically secure.  It is most
important that the chaff data does not contain any patterns
that can be used to discern it from wheat data without running 
the MAC.
\end{methoddesc}

\section{Crypto.PublicKey: Public Key Algorithms}
So far, the encryption algorithms described have all been \dfn{private
key} ciphers.  That is, the same key is used for both encryption and
decryption, so all correspondents must know it.  This poses a problem:
you may want encryption to communicate sensitive data over an insecure
channel, but how can you tell your correspondent what the key is?  You
can't just e-mail it to her because the channel is insecure.  One
solution is to arrange the key via some other way: over the phone or
by meeting in person.

Another solution is to use \dfn{public key} cryptography.  In a public
key system, there are two different keys: one for encryption and one for
decryption.  The encryption key can be made public by listing it in a
directory or mailing it to your correspondent, while you keep the
decryption key secret.  Your correspondent then sends you data encrypted
with your public key, and you use the private key to decrypt it.  While
the two keys are related, it's very difficult to derive the private key
given only the public key; however, deriving the private key is always
possible given enough time and computing power.  This makes it very
important to pick keys of the right size: large enough to be secure, but
small enough to be applied fairly quickly.

Many public key algorithms can also be used to sign messages; simply
run the message to be signed through a decryption with your private
key key.  Anyone receiving the message can encrypt it with your
publicly available key and read the message.  Some algorithms do only
one thing, others can both encrypt and authenticate.

The currently available public key algorithms are listed in the
following table:

\begin{tableii}{c|l}{}{Algorithm}{Capabilities}
\lineii{RSA}{Encryption, authentication/signatures}
\lineii{ElGamal}{Encryption, authentication/signatures}
\lineii{DSA}{Authentication/signatures}
\lineii{qNEW}{Authentication/signatures}
\end{tableii}

Many of these algorithms are patented.  Before using any of them in a
commercial product, consult a patent attorney; you may have to arrange
a license with the patent holder.

An example of using the RSA module to sign a message:
\begin{verbatim}
>>> from Crypto.Hash import MD5
>>> from Crypto.PublicKey import RSA
>>> RSAkey=RSA.generate(384, randfunc)   # This will take a while...
>>> hash=MD5.new(plaintext).digest()
>>> signature=RSAkey.sign(hash, "")
>>> signature   # Print what an RSA sig looks like--you don't really care.
('\021\317\313\336\264\315' ...,)
>>> RSAkey.verify(hash, signature)     # This sig will check out
1
>>> RSAkey.verify(hash[:-1], signature)# This sig will fail
0
\end{verbatim}

       
Public key modules make the following functions available:

\begin{funcdesc}{construct}{tuple}
Constructs a key object from a tuple of data.  This is
algorithm-specific; look at the source code for the details.  (To be
documented later.)
\end{funcdesc}

\begin{funcdesc}{generate}{size, randfunc, progress_func=\code{None}}
Generate a fresh public/private key pair.  \var{size} is a
algorithm-dependent size parameter; the larger it is, the more
difficult it will be to break the key.  Safe key sizes vary from
algorithm to algorithm; you'll have to research the question and
decide on a suitable key size for your application.  \code{randfunc}
is a random number generation function; it should accept a single
integer \var{N} and return a string of random data \var{N} bytes long.
You should always use a cryptographically secure random number
generator, such as the one defined in the \code{randpool} module;
\emph{don't} just use the current time and the \code{whrandom} module.

\var{progress_func} is an optional function that will be called with a short
string containing the key parameter currently being generated; it's
useful for interactive applications where a user is waiting for a key to
be generated.
\end{funcdesc}

If you want to interface with some other program, you will have to know
the details of the algorithm being used; this isn't a big loss.  If you
don't care about working with non-Python software, simply use the
\code{pickle} module when you need to write a key or a signature to a
file.  It's portable across all the architectures that Python supports,
and it's simple to use.

Public key objects always support the following methods.  Some of them
may raise exceptions if their functionality is not supported by the
algorithm.

\begin{funcdesc}{canencrypt}{}
Returns true if the algorithm is capable of encrypting and decrypting
data; returns false otherwise.  To test if a given key object can sign
data, use \code{key.canencrypt() and key.hasprivate()}.
\end{funcdesc}

\begin{funcdesc}{cansign}{}
Returns true if the algorithm is capable of signing data; returns false
otherwise.  To test if a given key object can sign data, use
\code{key.cansign() and key.hasprivate()}.
\end{funcdesc}

\begin{funcdesc}{decrypt}{tuple}
Decrypts \var{tuple} with the private key, returning another string.
This requires the private key to be present, and will raise an exception
if it isn't present.  It will also raise an exception if \var{string} is
too long.
\end{funcdesc}

\begin{funcdesc}{encrypt}{string, K}
Encrypts \var{string} with the private key, returning a tuple of
strings; the length of the tuple varies from algorithm to algorithm.  
\var{K} should be a string of random data that is as long as
possible.  Encryption does not require the private key to be present
inside the key object.  It will raise an exception if \var{string} is
too long.  For ElGamal objects, the value of \var{K} expressed as a
big-endian integer must be relatively prime to \code{self.p-1}; an
exception is raised if it is not.
\end{funcdesc}

\begin{funcdesc}{hasprivate}{}
Returns true if the key object contains the private key data, which
will allow decrypting data and generating signatures.
Otherwise this returns false.
\end{funcdesc}

\begin{funcdesc}{publickey}{}
Returns a new public key object that doesn't contain the private key
data. 
\end{funcdesc}

\begin{funcdesc}{sign}{string, K}
Sign \var{string}, returning a signature, which is just a tuple; in
theory the signature may be made up of any Python objects at all; in
practice they'll be either strings or numbers.  \var{K} should be a
string of random data that is as long as possible.  Different algorithms
will return tuples of different sizes.  \code{sign()} raises an
exception if \var{string} is too long.  For ElGamal objects, the value
of \var{K} expressed as a big-endian integer must be relatively prime to
\code{self.p-1}; an exception is raised if it is not.
\end{funcdesc}

\begin{funcdesc}{size}{}
Returns the maximum size of a string that can be encrypted or signed,
measured in bits.  String data is treated in big-endian format; the most
significant byte comes first.  (This seems to be a \emph{de facto} standard
for cryptographical software.)  If the size is not a multiple of 8, then
some of the high order bits of the first byte must be zero.  Usually
it's simplest to just divide the size by 8 and round down.
\end{funcdesc}

\begin{funcdesc}{verify}{string, signature}
Returns true if the signature is valid, and false otherwise.
\var{string} is not processed in any way; \code{verify} does
not run a hash function over the data, but you can easily do that yourself.
\end{funcdesc}

\subsection{The ElGamal and DSA algorithms}
For RSA, the \var{K} parameters are unused; if you like, you can just
pass empty strings.  The ElGamal and DSA algorithms require a real
\var{K} value for technical reasons; see Schneier's book for a detailed
explanation of the respective algorithms.  This presents a possible
hazard that can  
inadvertently reveal the private key.  Without going into the
mathematical details, the danger is as follows. \var{K} is never derived
or needed by others; theoretically, it can be thrown away once the
encryption or signing operation is performed.  However, revealing
\var{K} for a given message would enable others to derive the secret key
data; worse, reusing the same value of \var{K} for two different
messages would also enable someone to derive the secret key data.  An
adversary could intercept and store every message, and then try deriving
the secret key from each pair of messages.

This places implementors on the horns of a dilemma.  On the one hand,
you want to store the \var{K} values to avoid reusing one; on the other
hand, storing them means they could fall into the hands of an adversary.
One can randomly generate \var{K} values of a suitable length such as
128 or 144 bits, and then trust that the random number generator
probably won't produce a duplicate anytime soon.  This is an
implementation decision that depends on the desired level of security
and the expected usage lifetime of a private key.  I cannot choose and
enforce one policy for this, so I've added the \var{K} parameter to the
\code{encrypt} and \code{sign} functions.  You must choose \var{K} by
generating a string of random data; for ElGamal, when interpreted as a
big-endian number (with the most significant byte being the first byte
of the string), \var{K} must be relatively prime to \code{self.p-1}; any
size will do, but brute force searches would probably start with small
primes, so it's probably good to choose fairly large numbers.  It might be
simplest to generate a prime number of a suitable length using the
\code{Crypto.Util.number} module.

\subsection{Security Notes for Public-key Algorithms}
Any of these algorithms can be trivially broken; for example, RSA can be
broken by factoring the modulus \emph{n} into its two prime factors.
This is easily done by the following code:

\begin{verbatim}
for i in range(2, n): 
    if (n%i)==0: print i, 'is a factor' ; break
\end{verbatim}


However, \emph{n} is usually a few hundred bits long, so this simple
program wouldn't find a solution before the universe comes to an end.
Smarter algorithms can factor numbers more quickly, but it's still
possible to choose keys so large that they can't be broken in a
reasonable amount of time.  For ElGamal and DSA, discrete logarithms are
used instead of factoring, but the principle is the same.

Safe key sizes depend on the current state of computer science and
technology.  At the moment, one can roughly define three levels of
security: low-security commercial, high-security commercial, and
military-grade.  For RSA, these three levels correspond roughly to 512,
768, and 1024 bit-keys.  For ElGamal and DSA, the key sizes should be
somewhat larger for the same level of security, around 768, 1024, and
1536 bits.

\section{Crypto.Util: Odds and Ends}
This chapter contains all the modules that don't fit into any of the
other chapters.  

\subsection{Crypto.Util.number}

This module contains various functions of number-theoretic functions.  

\begin{funcdesc}{GCD}{x,y}
Return the greatest common divisor of \var{x} and \var{y}.
\end{funcdesc}

\begin{funcdesc}{getPrime}{N, randfunc}
Return an \var{N}-bit random prime number, using random data obtained
from the function \var{randfunc}.  \var{randfunc} must take a single
integer argument, and return a string of random data of the
corresponding length; the \code{getBytes()} method of a
\code{RandomPool} object will serve the purpose nicely, as will the
\code{read()} method of an opened file such as \file{/dev/random}.
\end{funcdesc}

\begin{funcdesc}{getRandomNumber}{N, randfunc}
Return an \var{N}-bit random number, using random data obtained from the
function \var{randfunc}.  As usual, \var{randfunc} must take a single
integer argument, and return a string of random data of the
corresponding length.
\end{funcdesc}

\begin{funcdesc}{inverse}{u, v}
Return the inverse of \var{u} modulo \var{v}.
\end{funcdesc}

\begin{funcdesc}{isPrime}{N}
Returns true if the number \var{N} is prime, as determined by a
Rabin-Miller test.
\end{funcdesc}

\index{random numbers}
\subsection{Crypto.Util.randpool}
For cryptographic purposes, ordinary random number generators are
frequently insufficient, because if some of their output is known, it is
frequently possible to derive the generator's future (or past) output.
This is obviously a Bad Thing; given the generator's state at some point
in time, someone could try to derive any keys generated using it.  The
solution is to use strong encryption or hashing algorithms to generate
successive data; this makes breaking the generator as difficult as
breaking the algorithms used.

\index{entropy}
Understanding the concept of \dfn{entropy} is important for using the
random number generator properly.  In the sense we'll be using it,
entropy measures the amount of randomness; the usual unit is in bits.
So, a single random bit has an entropy of 1 bit; a random byte has an
entropy of 8 bits.  Now consider a one-byte field in a database containing a
person's sex, represented as a single character \samp{M} or \samp{F}.
What's the entropy of this field?  Since there are only two possible
values, it's not 8 bits, but one; if you were trying to guess the value,
you wouldn't have to bother trying \samp{Q} or \samp{@}.  

Now imagine running that single byte field through a hash function that
produces 128 bits of output.  Is the entropy of the resulting hash value
128 bits?  No, it's still just 1 bit.  The entropy is a measure of how many
possible states of the data exist.  For English
text, the entropy of a five-character string is not 40 bits; it's
somewhat less, because not all combinations would be seen.  \samp{Guido}
is a possible string, as is \samp{In th}; \samp{zJwvb} is not.

The relevance to random number generation?  We want enough bits of
entropy to avoid making an attack on our generator possible.  An
example: One computer system had a mechanism which generated nonsense
passwords for its users.  This is a good idea, since it would prevent
people from choosing their own name or some other easily guessed string.
Unfortunately, the random number generator used only had 65536 states,
which meant only 65536 different passwords would ever be generated, and
it was easily to compute all the possible passwords and try them.  The
entropy of the random passwords was far too low.  By the same token, if
you generate an RSA key with only 32 bits of entropy available, there
are only about 4.2 billion keys you could have generated, and an
adversary could compute them all to find your private key.  See RFC 1750:
"Randomness Recommendations for Security" for an interesting discussion
of the issues related to random number generation.

The \code{randpool} module implements a strong random number generator
in the \code{RandomPool} class.  The internal state consists of a string
of random data, which is returned as callers request it.  The class
keeps track of the number of bits of entropy left, and provides a function to
add new random data; this data can be obtained in various ways, such as
by using the variance in a user's keystroke timings.  

\begin{funcdesc}{RandomPool}{\optional{numbytes, cipher, hash} }
An object of the \code{RandomPool} class can be created without
parameters if desired.  \var{numbytes} sets the number of bytes of
random data in the pool, and defaults to 160 (1280 bits). \var{hash}
can be a string containing the module name of the hash function to use
in stirring the random data, or a module object supporting the hashing
interface.  The default action is to use SHA.

The \var{cipher} argument is vestigial; it was removed from version
1.1 so RandomPool would work even in the limited exportable subset of
the code.  It can have any value at all, since it's no longer used at
all.

\end{funcdesc}

\code{RandomPool} objects define the following variables and methods:

\begin{funcdesc}{addEvent}{time\optional{, string}}
Adds an event to the random pool.  \var{time} should be set to the
current system time, measured at the highest resolution available.
\var{string} can be a string of data that will be XORed into the pool,
and can be used to increase the entropy of the pool.  For example, if
you're encrypting a document, you might use the hash value of the
document; an adversary presumably won't have the plaintext of the
document, and thus won't be able to use this information to break the
generator.
\end{funcdesc}

The return value is the value of \code{self.entropy} after the data has
been added.  The function works in the following manner: the time
between successive calls to the \code{addEvent} method is determined,
and the entropy of the data is guessed; the larger the time between
calls, the better.  The system time is then read and added to the pool,
along with the \var{string} parameter, if present.  The hope is that the
low-order bits of the time are effectively random.  In an application,
it is recommended that \code{addEvent()} be called as frequently as
possible, with whatever random data can be found.

\begin{datadesc}{bits}
A constant integer value containing the number of bits of data in
the pool, equal to the \code{bytes} variable multiplied by 8.
\end{datadesc}

\begin{datadesc}{bytes}
A constant integer value containing the number of bytes of data in
the pool.
\end{datadesc}

\begin{datadesc}{entropy}
An integer value containing the number of bits of entropy currently in
the pool.  The value is incremented by the \code{addEvent()} method,
and decreased by the \code{getBytes} method.
\end{datadesc}

\begin{funcdesc}{getBytes}{num}
Returns a string containing \var{num} bytes of random data, and
decrements the amount of entropy available.  It is not an error to
reduce the entropy to zero, or to call this function when the entropy
is zero.  This simply means that, in theory, enough random information has been
extracted to derive the state of the generator.  It is the caller's
responsibility to monitor the amount of entropy remaining and decide
whether it is sufficent for secure operation.
\end{funcdesc}

\begin{funcdesc}{stir}{}
Scrambles the random pool using the previously chosen encryption and
hash function.  An adversary may attempt to learn or alter the state
of the pool in order to affect its future output; this function
destroys the existing state of the pool in a non-reversible way.  It
is recommended that \code{stir()} be called before and after using
the \code{RandomPool} object.  Even better, several calls to
\code{stir()} can be interleaved with calls to \code{addEvent()}.
\end{funcdesc}

The \code{KeyboardRandomPool} class is a subclass of \code{RandomPool} 
that adds the capability to save and load the pool from a disk file, and
provides a method to obtain random data from the keyboard.

\begin{funcdesc}{KeyboardRandomPool}{\optional{filename, numbytes, cipher, hash}}
The path given in \var{filename} will be automatically opened, and an
existing random pool read; if no such file exists, the pool will be
initialized as usual.  If omitted, the filename defaults to the empty
string, which will prevent it from being saved to a file.  The other
arguments are identical to those for the \code{RandomPool} constructor.
\end{funcdesc}

\begin{funcdesc}{randomize}{}
(Unix systems only)  Obtain random data from the keyboard.  This works
by prompting the
user to hit keys at random, and then using the keystroke timings (and
also the actual keys pressed) to add entropy to the pool.  This works
similarly to PGP's random pool mechanism.
\end{funcdesc}

\begin{funcdesc}{save}{}
Opens the file named by the \code{filename} attribute, and saves the
random data into the file using the \code{pickle} module.
\end{funcdesc}


\subsection{Crypto.Util.RFC1751}
The keys for private-key algorithms should be arbitrary binary data.
Many systems err by asking the user to enter a password, and then using
the password as the key.  This limits the space of possible keys, as
each key byte is constrained within the range of possible ASCII
characters, 32-127, instead of the whole 0-255 range possible with ASCII.
Unfortunately, it's difficult for humans to remember 16 or 32 hex
digits.  

One solution is to request a lengthy passphrase from the user, and then
run it through a hash function such as SHA or MD5.  Another solution is
discussed in RFC 1751, "A Convention for Human-Readable 128-bit Keys",
by Daniel L. McDonald.  Binary keys are transformed into a list of short
English words that should be easier to remember.  For example, the hex
key EB33F77EE73D4053 is transformed to "TIDE ITCH SLOW REIN RULE MOT".

\begin{funcdesc}{Key2English}{key}
Accepts a string of arbitrary data \var{key}, and returns a string
containing uppercase English words separated by spaces.  \var{key}'s
length must be a multiple of 8.
\end{funcdesc}

\begin{funcdesc}{English2Key}{string}
Accepts \var{string} containing English words, and returns a string of
binary data representing the key.  Words must be separated by
whitespace, and can be any mixture of uppercase and lowercase
characters.  6 words are required for 8 bytes of key data, so
the number of words in \var{string} must be a multiple of 6.
\end{funcdesc}

\section{The Demonstration Programs}
The Python cryptography modules comes with various demonstration
programs, located in the \file{Demo/} directory.  None of them is
particularly well-finished, or suitable for serious use.  Rather,
they're intended to illustrate how the toolkit is used, and to provide
some interesting possible uses.  Feel free to incorporate the code (or
modifications of it) into your own programs.

\subsection{Demo 1: \file{cipher}}

\index{crypt}
\index{cipher (demo program)}
\index{Enigma}
\index{Crypt Breaker's Workbench}
\file{cipher} encrypts and decrypts files.  On most Unix systems, the
\file{crypt} program uses a variant of the Enigma cipher.  This is not
secure, and there exists a freely available program called ``Crypt
Breaker's Workbench'' which helps in breaking the cipher if you have
some knowledge of the encrypted data.

\file{cipher} is a more secure file encryption program.  Simply list
the names of the files to be encrypted on the command line.
\file{cipher} will go through the list and encrypt or decrypt them;
\file{cipher} can recognize files it has previously encrypted.  The
ciphertext of a file is placed in a file of the same name with
'\samp{.cip}' appended; the original file is not deleted, since I'm
not sure that all errors during operation are caught, and I don't want
people to accidentally erase important files.

There are two command-line options: \code{-c} and \code{-k}.  Both of
them require an argument.  \code{-c \var{ciphername}} uses the
given encryption algorithm \var{ciphername}; for example,
\code{-c des} will use the DES algorithm.  The name should be the same
as an available module name; thus it should be in lowercase letters.
The default cipher is IDEA.

\index{Linux}
\index{cipher}
\code{-k \var{key}} can be used to set the encryption key to be
used.  Note that on a multiuser Unix system, the \code{ps} command can
be used to view the arguments of commands executed by other users, so
this is insecure; if you're the only user (say, on your home computer
running Linux) you don't have to worry about this.  If no key is set
on the command line, \file{cipher} will prompt the user to input a key
on standard input.

\subsubsection{Technical Details}

The encrypted file is not pure ciphertext.  First comes a magic
string; this is currently the sequence \samp{ctx} and a byte
containing 1 (the version number of \file{cipher}).
This is followed by the null-terminated name of the encryption
algorithm, and the rest of the file contains the ciphertext.  

\index{feedback mode, CBC}
The plaintext is encrypted in CBC mode.  The initial value for the
feedback is always set to a block filled with the letter 'A', and then
a block of random data is encrypted.  This garbage block will be
discarded on decryption.  Note that the random data is not generated
in a cryptographically secure way, and this may provide a tiny foothold for
an attacker.

After the random block is generated, the magic string, length of the
original file, and original filename are all encrypted before the file
data is finally processed.  Some extra characters of padding may be
added to obtain an integer number of blocks.  This padding will also
be discarded on decryption.  Note that the plaintext file will be
completely read into memory before encryption is performed; no
buffering is done.  Therefore, don't encrypt 20-megabyte files unless
you're willing to face the consequences of a 20-megabyte process.

Areas for improvements to \file{cipher} are: cryptographically secure
generation of random data
for padding, key entry, and buffering of file
input.

\subsection{Demo 2: \file{secimp} and \file{sign}}

\file{secimp} demonstrates an application of the Toolkit that may be
useful if Python is being used as an extension language for mail and Web
clients: secure importing of Python modules.  To use it, run
\file{sign.py} in a directory with several compiled Python files
present.  It will use the key in \file{testkey.py} to generate digital
signatures for the compiled Python code, and save both the signature and
the code in a file ending in \samp{.pys}.  Then run \code{python -i
secimp.py}, and import a file by using \code{secimport}.  

For example, if \file{foo.pys} was constructed, do
\code{secimport('foo')}.  The import should succeed.  Now fire up Emacs
or some other editor, and change a string in the code in \file{foo.pys};
you might try changing a letter in the name of a variable.  When you run
\code{secimport('foo')}, it should raise an exception reporting the
failed signature.  If you execute the statement \code{__import__ =
secimport}, the secure import will be used by default for all future
module imports.  Alternatively, if you were creating a restricted
execution environment using \file{rexec.py}, you could place
\code{secimport()} in the restricted environment's namespace as the
default import function.

\section{Extending the cryptography modules}
Preserving the a common interface for cryptographic routines is a
good idea.  This chapter
explains how to interface your own routines to the Toolkit.

The basic process is as follows:
\begin{enumerate}
\item  Modify the default definition of a C structure to include
whatever instance data your algorithm requires.
\item  Write 3 or 4 standard routines.  Their names and parameters are
specified in the following subsections.
\item  Modify \file{buildkit} to contain an entry for your new
algorithm.  Then run \file{buildkit} to rebuild all the source files. 
\item  Send a copy of the code to me, if you like; code for new
algorithms will be gratefully accepted.
\end{enumerate}

\subsection{Creating a Custom Object}
In the C code for the interpreter, Python objects are defined as a
structure.  The default structure is the following:
\begin{verbatim}
typedef struct 
{
 PCTObject_HEAD
} ALGobject;
\end{verbatim}


\code{PCTObject_HEAD} is a preprocessor macro which will contain various
internal variables used by the interpreter; it must always be the
first item in the structure definition, and must not be followed by a
semicolon.  Following it, you can put whatever instance variables you
require.  Data that does not depend on the instance or key, such as a
static lookup table, need not be encapsulated inside objects; instead,
it can be defined as a variable interior to the module.

As an example, for IDEA encryption, a schedule of encryption and
decryption data has to be maintained, resulting in the following
definition:
\begin{verbatim}
typedef struct 
{
 PCTObject_HEAD
 int EK[6][9], DK[6][9];
} IDEAobject;
\end{verbatim}


\subsection{Standard Routines}

\index{buildkit}
The interface to Python is implemented in the files ending in
\samp{.in}, so \file{hash.in} contains the basic code for modules
containing hash functions, for example.  \file{buildkit}, a Python
script, reads the configuration file and generates source code by
interweaving the interface files and the implementation file.

\index{buildkit}
If your algorithm is called ALG, the implementation should be in the
file \file{ALG.c}. This is case-sensitive, as are the following function
names.  

\subsubsection{Hash functions}

\begin{itemize}
\item \code{void \var{ALG}init(\var{ALG}object *self);}
\item \code{void \var{ALG}update(\var{ALG}object *self, char *buffer, int length);}
\item \code{PyObject *\var{ALG}digest(\var{ALG}object *self);}
\item \code{void \var{ALG}copy(\var{ALG}object *source, \var{ALG}object *dest);}
\end{itemize}

\begin{funcdesc}{void ALGinit}{\rm ALGobject *\var{self}}
\index{ALGinit}
This function should initialize the hashing object, setting 
state variables to their expected initial state.
\end{funcdesc}

\begin{funcdesc}{void ALGupdate}{\rm ALGobject *\var{self}, 
char *\var{buffer}, int \var{length}}
\index{ALGupdate}
This function should perform a hash on the region pointed to by
\var{buffer}, which will contain \var{length} bytes.  The contents of
the object pointed to by \var{self} should be updated appropriately. 
\end{funcdesc}

\begin{funcdesc}{void ALGdigest}{\rm ALGobject *\var{self}}
This function returns a string containing the value of the hash
function.  The object should not be changed in any way by this
function.  Some hash functions require some computation to be
performed before returning a value; for example, the number of bytes
may be hashed into the final value.  If this is the case for your hash
function, you must make a copy of the object's data, perform the final
computation on that copy, and return the result.
\end{funcdesc}

Results are returned by calling a Python function,
\code{PyString_FromStringAndSize(char *\var{string}, int \var{length})}.  This
function returns a string object which should be returned to the
caller.  So, the last line of the \code{ALGdigest}
function might be:
\begin{verbatim}
  return PyString_FromStringAndSize(digest, 16);
\end{verbatim}

\index{ALGdigest}

\begin{funcdesc}{void ALGcopy}{\rm ALGobject *\var{source}, ALGobject *\var{dest}}
Given the source and destination objects, the state variables of the
\var{source} object should be copied to the \var{dest} object; the
source object should not be altered in any way by the operation.
\index{ALGcopy}
\end{funcdesc}

\subsubsection{Block ciphers}
\begin{itemize}
\item \code{void ALGinit(ALGobject *\var{self}, unsigned char *\var{key}, int \var{length});}
\item \code{PyObject *ALGencrypt(ALGobject *\var{self}, unsigned char *\var{block});}
\item \code{PyObject *ALGdecrypt(ALGobject *\var{self}, unsigned char *\var{block});}
\end{itemize}

\begin{funcdesc}{void ALGinit}{\rm ALGobject *\var{self}, unsigned char *\var{key}, int \var{length}}
This function initializes a block cipher object to encrypt and decrypt
with \var{key}.  If the cipher requires a fixed-length key, then the
buffer pointed to by \var{key} will always of that length, and the
value of \var{length} will be a random value that should be ignored.
If the algorithm accepts a variable-length key, then \var{length} will
be nonzero, and will contain the size of the key.
\index{ALGinit}
\end{funcdesc}

\begin{funcdesc}{void ALGencrypt}{\rm ALGobject *\var{self}, unsigned char *\var{block}}
This function should encrypt the data pointed to by \var{block}, using
the key-dependent data contained in \var{self}.  Only ECB mode needs
to be implemented; \code{block.in} takes care of the other
ciphering modes.
\index{ALGencrypt}
\end{funcdesc}

\begin{funcdesc}{void ALGdecrypt}{\rm ALGobject *\var{self}, unsigned char *\var{block}}
This function should decrypt the data pointed to by \var{block}, using
the key-dependent data contained in \var{self}.
\index{ALGdecrypt}
\end{funcdesc}

\subsection{Portability macros}

Implementation code must be carefully written to produce the same
results with any machine or compiler, without having to set any
compile-time definitions.  Code that is simply portable by nature is
preferable, but it is possible to detect features of the host machine
when new objects are created, and then execute special code to convert
data to a preferred form.

While portability macros are written for speed, there's no need to
execute them on every encryption or updating operation.  Instead, add
variables to your object to hold the values of the portability macros,
and execute the macros only once per object, in your
\code{ALGinit} function.  Then the code can simply check the
results of the macros and act appropriately.

Currently there is only one portability macro defined:

\begin{funcdesc}{void TestEndianness}{variable}
Determines the endianness of the current machine, and sets
\var{variable} to a constant representing the value for this machine.
Possible constants are \code{PCT_BIG_ENDIAN} and \code{PCT_LITTLE_ENDIAN};
they are defined along with the \code{TestEndianness} macro.
\end{funcdesc}

\subsection{Informing the author}
Code for additional cryptographic algorithms can be mailed to me at
\email{akuchling@acm.org}.  You can make things much easier for me by doing the
following:
\begin{itemize}
\item  If you wrote the code, please release it into the public domain
or under the MIT X11 license.  If you didn't write it, please tell me where
to find the original author or source code, so that I can check the
licensing conditions.
\item  Include some test data.  It is not sufficient to check that
encryption and decryption cancel out properly.  An implementation
might work fine on a single platform, but two machines with
different endianness might produce different results.  This would be
fatal for portability and interoperating programs.  So, please include
test data; you can either send me patches to \file{test.py}, or simply
send me documents describing the data.
\end{itemize}

\end{document}
